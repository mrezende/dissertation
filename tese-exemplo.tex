% Arquivo LaTeX de exemplo de dissertação/tese a ser apresentados ?CPG do IME-USP
% 
% Vers? 5: Sex Mar  9 18:05:40 BRT 2012
%
% Cria?o: Jes?s P. Mena-Chalco
% Revis?: Fabio Kon e Paulo Feofiloff
%  
% Obs: Leia previamente o texto do arquivo README.txt

\documentclass[12pt,twoside,a4paper]{book} % utilizando extbook para atender ao modelo do IPT

\usepackage{sectsty}
\allsectionsfont{\Large\bfseries}

% ---------------------------------------------------------------------------- %
% Pacotes 
\usepackage{fontspec}
\setmainfont{Arial} % funciona com windows/mac
\usepackage{polyglossia}
\setdefaultlanguage{brazil}
%\usepackage[bf,small,compact]{titlesec} % cabe?lhos dos t?ulos: menores e compactos
%\usepackage[fixlanguage]{babelbib}

\usepackage[xetex]{graphicx}           % usamos arquivos pdf/png como figuras
\usepackage{setspace}                   % espa?mento flex?el
\usepackage{indentfirst}                % indenta?o do primeiro par?rafo
\usepackage{makeidx}                    % ?dice remissivo
\usepackage[nottoc]{tocbibind}          % acrescentamos a bibliografia/indice/conteudo no Table of Contents
\usepackage{courier}                    % usa o Adobe Courier no lugar de Computer Modern Typewriter
\usepackage{type1cm}                    % fontes realmente escal?eis
\usepackage{listings}                   % para formatar c?igo-fonte (ex. em Java)
\usepackage{titletoc}
\usepackage[font=small,format=plain,labelfont=bf,up,textfont=it,up]{caption}
\usepackage[usenames,svgnames,dvipsnames]{xcolor}
\usepackage[a4paper,top=2.54cm,bottom=2.0cm,left=2.0cm,right=2.54cm]{geometry} % margens
%\usepackage[pdftex,plainpages=false,pdfpagelabels,pagebackref,colorlinks=true,citecolor=black,linkcolor=black,urlcolor=black,filecolor=black,bookmarksopen=true]{hyperref} % links em preto
\usepackage[plainpages=false,pdfpagelabels,pagebackref,colorlinks=true,citecolor=DarkGreen,linkcolor=NavyBlue,urlcolor=DarkRed,filecolor=green,bookmarksopen=true]{hyperref} % links coloridos
\usepackage[all]{hypcap}                    % soluciona o problema com o hyperref e capitulos
\usepackage[round,sort,nonamebreak]{natbib} % cita?o bibliogr?ica textual(plainnat-ime.bst)
\fontsize{60}{62}\usefont{OT1}{cmr}{m}{n}{\selectfont}



% todo-notes
\usepackage[colorinlistoftodos]{todonotes}


\usepackage[acronym, toc, xindy]{glossaries}
\usepackage{amssymb}
\usepackage{multirow}
\usepackage{bm}

\usepackage{longtable}

\usepackage{array}
\newcolumntype{P}[1]{>{\raggedleft\arraybackslash}p{#1}}



\makenoidxglossaries
\newglossaryentry{ml}
{
    name=aprendizagem de máquina,
    description={sistema ou programa que constrói um modelo preditivo a partir de dados de entrada \citep{glossary-ml}}
}

\newglossaryentry{modelo}
{
    name=modelo,
    description={Representação do que um sistema de \gls{ml} aprendeu a partir de dados de treinamento \citep{glossary-ml}}
}

\newglossaryentry{sof}
{
    name=StackOverflow,
    description={Site de perguntas e respostas de programação. Endereço do site: \url{https://www.stackoverflow.com/}}
}

\newglossaryentry{git}
{
    name=git,
    description={git é um versionador de controle distribuído para rastrear alterações no código-fonte durante o desenvolvimento de software. Endereço do site: \url{https://git-scm.com/} \citep{wikipedia-git-2019}}
}

\newglossaryentry{github}
{
    name=GitHub,
    description={GitHub é uma plataforma de hospedagem de código-fonte com controle de versão usando o \gls{git}. Endereço do site: \url{https://www.github.com/}}
}

\newglossaryentry{representacao-distribuida}
{
    name=representação distribuída,
    description={Representação distribuída significa uma relação de muitos para muitos entre dois tipos de representações (por exemplo, conceitos e \gls{neuron}s) \citep{Hinton-distributed-representatons:1986}. 
    \begin{itemize}
        \item Cada conceito é representado por muitos \gls{neuron}s
        \item Cada \gls{neuron} participa na representação de muitos conceitos
    \end{itemize}
    }
}

\newglossaryentry{one-hot-encoding}
{
    name=\textit{one-hot encoding},
    description={\textit{one-hot encoding} é um vetor esparso que contém:
    \begin{itemize}
        \item Um elemento cujo valor é definido como 1
        \item O restante dos elementos tem o valor definido como 0
    \end{itemize}
    \textit{One-hot enconding} normalmente é utilizado para representar palavras e ou atributos que contém uma quantidade finita de valores \citep{glossary-ml}
    }
}

\newglossaryentry{neuron}
{
    name=neurônio,
    description={Um neurônio é um nó numa rede neural, que tipicamente recebe múltiplos valores de entrada e gera um valor de resultado. O neurônio aplica uma função de ativação (transformação não-linear) na soma dos valores de entrada com seus respectivos pesos \citep{glossary-ml}}
}

\newglossaryentry{bag-of-words}
{
    name=bag of words,
    description={Vetor composto por palavras indiferente à ordem e permutação. Ver exemplo na Seção~\ref{sec:representacao-das-sentencas-fundamentacao-teorica}}
}

\newglossaryentry{mecanismo-atencao}{
name=mecanismo de atenção,
description={
    O mecanismo de atenção calcula a média ponderada dos elementos de um vetor e o principal objetivo dele é encontrar um peso para cada elemento. Ao aplicarmos o mecanismo em uma tarefa de tradução, por exemplo, a cada momento que uma palavra é traduzida, o mecanismo de atenção foca em partes diferentes da sentença, i.e, ele aprende a ''prestar atenção'' nas palavras mais relevantes \citep{Goodfellow-et-al-2016}
}}

\newglossaryentry{docstring}{
name=docstring,
description={
    Em programação, um \textit{docstring} é um texto especificado no código-fonte que é usado para documentar um trecho específico do código \cite{wikipedia-docstring-2019}
}}

\newglossaryentry{jupyter}{
name=Jupyter,
description={
    \textit{Jupyter Notebook} é uma ferramenta interativa que permite desevolver, executar e documentar código em uma aplicação web. O termo \textit{notebook} refere-se a um caderno de anotações, pois é possível desenvolver, salvar as saídas do programa e fazer anotações \cite{jupyter-2019}
}}

\newglossaryentry{colab}{
name=Colab,
description={
    É uma ferramenta de pesquisa e ensino para aprendizagem de máquina. É um ambiente \Gls{jupyter} que não necessita configuração ou instalação \cite{colab-2019}
}}

\newglossaryentry{unif}{
name=Unif,
description={
    Arquitetura de rede neural com mecanismo de atenção proposta por \cite{cambronero-deep-learning-code-search:2019} para a recuperação de trecho de código-fonte
}}

\newglossaryentry{xlnet}{
name=XLNet,
description={
    Arquitetura autoregressiva para compreensão de linguagem durante o pré-treinamento \citep{yang2019xlNet}
}}

\newglossaryentry{word2vec}{
name=Word2vec,
description={
    Word2vec é uma técnica para representar palavras através de vetores de \gls{representacao-distribuida}. Mais informações na Seção~\ref{sec:fundamentao-representacao-tokens-palavras}
}}

\newglossaryentry{tensorflow}{
name=Tensorflow,
description={
    Tensorflow é uma biblioteca aberta de aprendizagem de máquina aplicável a uma ampla variedade de tarefas \citep{wikipedia-tensorflow-2020}
}}

\newglossaryentry{keras}{
name=Keras,
description={
    Keras é uma biblioteca aberta de redes neurais escrita em Python \citep{wikipedia-keras-2020}
}}




\newacronym{ide}{IDE}{Integrated Development Environment}

\newacronym{rnn}{RNN}{Recurrent Neural Network}
\newacronym{lstm}{LSTM}{Long Short Term Memory}
\newacronym{nlp}{NLP}{Processamento de Linguagem Natural}
\newacronym{vae}{VAE}{Variational AutoEncoder}
\newacronym{tf-idf}{TFIDF}{Term Frequency–Inverse Document
Frequency}
\newacronym{cbow}{CBoW}{Comsuption Bag of Words}
\newacronym{sof-ab}{SO}{Stack Overflow}
\newacronym{github-ab}{GH}{GitHub}
\newacronym{cnn}{CNN}{Convolutional Neural Network}
\newacronym{vem}{VEM}{Workshop on Software Visualization, Evolution and Maintenance}
\newacronym{mrr}{MRR}{Mean Reciprocal Rank}
\newacronym{vgpu}{vGPU}{Virtual Graphics Processing Unit}
\newacronym{elmo}{ELMo}{Embeddings from Language Models}
\newacronym{bert}{BERT}{Bidirection Encoder Representations from Transformers}
\newacronym{squad}{SQuAD}{Stanford Question Answering Dataset}
\newacronym{map}{MAP}{Mean Average Precision}
\newacronym{ndcg}{NDCG}{Normalized Discounted Cumulative Gain}
\newacronym{gpu}{GPU}{Graphics Processing Unit}
\newacronym{cpu}{CPU}{Central Processing Unit}












% ---------------------------------------------------------------------------- %
% Cabe?lhos similares ao TAOCP de Donald E. Knuth
\usepackage{fancyhdr}
\pagestyle{fancy}
\fancyhf{}
\renewcommand{\chaptermark}[1]{\markboth{\MakeUppercase{#1}}{}}
\renewcommand{\sectionmark}[1]{\markright{\MakeUppercase{#1}}{}}
\renewcommand{\headrulewidth}{0pt}

% ---------------------------------------------------------------------------- %
\graphicspath{{./figuras/}}             % caminho das figuras (recomend?el)
\frenchspacing                          % arruma o espa?: id est (i.e.) e exempli gratia (e.g.) 
\urlstyle{same}                         % URL com o mesmo estilo do texto e n? mono-spaced
\makeindex                              % para o ?dice remissivo
\raggedbottom                           % para n? permitir espa?s extra no texto
\fontsize{60}{62}\usefont{OT1}{cmr}{m}{n}{\selectfont}
\cleardoublepage
\normalsize

% ---------------------------------------------------------------------------- %
% Op?es de listing usados para o c?igo fonte
% Ref: http://en.wikibooks.org/wiki/LaTeX/Packages/Listings
\lstset{ %
language=Java,                  % choose the language of the code
basicstyle=\footnotesize,       % the size of the fonts that are used for the code
numbers=left,                   % where to put the line-numbers
numberstyle=\footnotesize,      % the size of the fonts that are used for the line-numbers
stepnumber=1,                   % the step between two line-numbers. If it's 1 each line will be numbered
numbersep=5pt,                  % how far the line-numbers are from the code
showspaces=false,               % show spaces adding particular underscores
showstringspaces=false,         % underline spaces within strings
showtabs=false,                 % show tabs within strings adding particular underscores
frame=single,	                % adds a frame around the code
framerule=0.6pt,
tabsize=2,	                    % sets default tabsize to 2 spaces
captionpos=b,                   % sets the caption-position to bottom
breaklines=true,                % sets automatic line breaking
breakatwhitespace=false,        % sets if automatic breaks should only happen at whitespace
escapeinside={\%*}{*)},         % if you want to add a comment within your code
backgroundcolor=\color[rgb]{1.0,1.0,1.0}, % choose the background color.
rulecolor=\color[rgb]{0.8,0.8,0.8},
extendedchars=true,
xleftmargin=10pt,
xrightmargin=10pt,
framexleftmargin=10pt,
framexrightmargin=10pt
}

\usepackage{color}

\definecolor{mygreen}{rgb}{0,0.6,0}
\definecolor{mygray}{rgb}{0.5,0.5,0.5}
\definecolor{mymauve}{rgb}{0.58,0,0.82}

\lstset{ %
  language=python,
  backgroundcolor=\color{white},   % choose the background color
  basicstyle=\footnotesize,        % size of fonts used for the code
  breaklines=true,                 % automatic line breaking only at whitespace
  captionpos=b,                    % sets the caption-position to bottom
  commentstyle=\color{mygreen},    % comment style
  escapeinside={\%*}{*)},          % if you want to add LaTeX within your code
  keywordstyle=\color{blue},       % keyword style
  stringstyle=\color{mymauve},     % string literal style
  numbers=none,
}

\renewcommand{\lstlistingname}{Trecho de código-fonte}% Listing -> Algorithm
\renewcommand{\lstlistlistingname}{Lista de trechos de códigos-fontes}% List of Listings -> List of Algorithms

% ---------------------------------------------------------------------------- %


\usepackage{minted}
\usepackage[most, minted]{tcolorbox}


\newtcblisting{mypython}[1]{%
listing engine=minted,
minted style=colorful,
minted language=python,
minted options= {escapeinside=||},
listing only,
title={#1}, fonttitle=\bfseries,
enlarge top by=1cm,%     equivalent to mdframed 'skipabove'
enlarge bottom by=1cm,%  equivalent to mdframed 'skipbelow'
  }
  
\newtcblisting{mypython-linenumber}[1]{%
listing engine=minted,
minted style=colorful,
minted language=python,
minted options= {xleftmargin=10pt, escapeinside=||, linenos},
listing only,
title={#1}, fonttitle=\bfseries,
enlarge top by=1cm,%     equivalent to mdframed 'skipabove'
enlarge bottom by=1cm,%  equivalent to mdframed 'skipbelow'
  }
  
\newtcblisting{mypythonembedding}[1]{%
listing engine=minted,
colback=green!5!white,colframe=green!75!black,
minted language=python,
listing only,
fonttitle=\bfseries,
enlarge top by=1cm,%     equivalent to mdframed 'skipabove'
enlarge bottom by=1cm,%  equivalent to mdframed 'skipbelow'
minted options= {escapeinside=||},
title={#1}
}

\newtcblisting{mypythongreen}[1]{%
listing engine=minted,
colback=green!5!white,colframe=green!75!black,
minted language=python,
listing only,
fonttitle=\bfseries,
%enlarge top by=1cm,%     equivalent to mdframed 'skipabove'
%enlarge bottom by=1cm,%  equivalent to mdframed 'skipbelow'
minted options= {escapeinside=||},
title={#1}
}

\newtcblisting{mypythonred}[1]{%
listing engine=minted,
colback=red!5!white,colframe=red!75!black,
minted language=python,
listing only,
fonttitle=\bfseries,
%enlarge top by=1cm,%     equivalent to mdframed 'skipabove'
%enlarge bottom by=1cm,%  equivalent to mdframed 'skipbelow'
minted options= {escapeinside=||},
title={#1}
}

\newtcblisting{mypython-without-margin}[1]{%
listing engine=minted,
minted style=colorful,
minted language=python,
minted options= {escapeinside=||},
listing only,
title={#1}, fonttitle=\bfseries
  }


% Corpo do texto
\begin{document}
\frontmatter 
% cabe?lho para as p?inas das se?es anteriores ao cap?ulo 1 (frontmatter)
\fancyhead[RO]{{\footnotesize\rightmark}\hspace{2em}\thepage}
\setcounter{tocdepth}{2}
\fancyhead[LE]{\thepage\hspace{2em}\footnotesize{\leftmark}}
\fancyhead[RE,LO]{}
\fancyhead[RO]{{\footnotesize\rightmark}\hspace{2em}\thepage}

\onehalfspacing  % espa?mento

% ---------------------------------------------------------------------------- %
% CAPA
% Nota: O t?ulo para as disserta?es/teses do IME-USP devem caber em um 
% orif?io de 10,7cm de largura x 6,0cm de altura que h?na capa fornecida pela SPG.
\thispagestyle{empty}
\begin{center}
    \large{\textbf{Instituto de Pesquisas Tecnológicas do Estado de São Paulo}}\\
    
    \vspace*{4cm}
    
    
    
    
    \large{\textbf{Marcelo de Rezende Martins}}
    
    \vspace*{6cm}
    
    \textbf{\large{Aprendizagem de representação através do uso de redes neurais convolucionais na recuperação de trecho de código-fonte}}\\
    
    
    
   \vspace*{10cm}
   
    \large{\textbf{São Paulo}} \\
    \large{\textbf{2019}}
\end{center}

% ---------------------------------------------------------------------------- %
% P?ina de rosto (S?PARA A VERS? DEPOSITADA - ANTES DA DEFESA)
% Resolu?o CoPGr 5890 (20/12/2010)
%
% IMPORTANTE:
%   Coloque um '%' em todas as linhas
%   desta p?ina antes de compilar a vers?
%   final, corrigida, do trabalho
%
%
\newpage
\thispagestyle{empty}
    \begin{center}
        Marcelo de Rezende Martins\\
        \vspace*{2.3 cm}
        Aprendizagem de representação através do uso de redes neurais convolucionais na recuperação de trecho de código-fonte\\
        \vspace*{2 cm}
    \end{center}

    \vskip 2cm

    \hspace{6cm}\begin{minipage}{0.48\linewidth}
	Dissertação de Mestrado apresentada
	ao Instituto de Pesquisas Tecnológicas do\\
	Estado de São Paulo - IPT, como 
	parte dos requisitos para a obtenção do 
	título de Mestre em Engenharia de 
	Computação
    \end{minipage}
    
    \vskip 2cm
    
    \hspace{6cm}\begin{minipage}{0.48\linewidth}
	Data da aprovação \rule{0.7cm}{0.4pt}/\rule{0.7cm}{0.4pt}/\rule{1.4cm}{0.4pt}
    \end{minipage}
    
    \vskip 2cm
    
    \hspace{6cm}\begin{minipage}{0.48\linewidth}
	\rule{7cm}{0.4pt}\\
	Prof. Dr. Marco Aurélio Gerosa\\
	Nothern Arizona University (NAU)\\
    \end{minipage}
\\
\\
Membros da Banca Examinadora:\\ 
\\
Prof. Dr. Marco Aurélio Gerosa (Orientador)\\
Nothern Arizona University (NAU)\\
\\
Prof. Dr. Marcelo Finger (Membro)\\
Instituto de Matemática e Estatística da Universidade de São Paulo (IME-USP)\\
\\
Prof. Dr. Marcelo Novaes de Rezende (Membro)\\
Mestrado Engenharia de Computação\\


\pagebreak


% ---------------------------------------------------------------------------- %
% P?ina de rosto (S?PARA A VERS? CORRIGIDA - AP? DEFESA)
% Resolu?o CoPGr 5890 (20/12/2010)
%
% Nota: O t?ulo para as disserta?es/teses do IME-USP devem caber em um 
% orif?io de 10,7cm de largura x 6,0cm de altura que h?na capa fornecida pela SPG.
%
% IMPORTANTE:
%   Coloque um '%' em todas as linhas desta
%   p?ina antes de compilar a vers? do trabalho que ser?entregue
%   ?Comiss? Julgadora antes da defesa
%
%
\newpage
\thispagestyle{empty}
    \begin{center}
        Marcelo de Rezende Martins\\
        \vspace*{2.3 cm}
        Aprendizagem de representação através do uso de redes neurais convolucionais na recuperação de trecho de código-fonte\\
        \vspace*{2 cm}
    \end{center}

    \vskip 2cm

    \hspace{6cm}\begin{minipage}{0.48\linewidth}
	Dissertação de Mestrado apresentada
	ao Instituto de Pesquisas Tecnológicas do\\
	Estado de São Paulo - IPT, como 
	parte dos requisitos para a obtenção do 
	título de Mestre em Engenharia de 
	Computação
    \end{minipage}
    
    \vskip 2cm
    
    \hspace{6cm}\begin{minipage}{0.48\linewidth}
	Área de Concentração: Engenharia de Software
    \end{minipage}
    
    \vskip 2cm
    
    \hspace{6cm}\begin{minipage}{0.48\linewidth}
	Orientador: Prof. Dr. Marco Aurélio Gerosa
    \end{minipage}
    
    \vskip 6cm
    
    \begin{center}
        São Paulo\\
        Março / 2020
    \end{center}

\pagebreak

\newpage
\thispagestyle{empty}

\begin{center}
        \vspace*{6 cm}
        \begin{quote}
            \textit{Sentir é criar.\\
            Sentir é pensar sem ideias, e por isso sentir é compreender, visto que o Universo não tem ideias.}
        \end{quote}
        \begin{flushright}
        Fernando Pessoa, Para Orpheu - Sentir é criar.\\
        \end{flushright}
    \end{center}


\pagebreak

\pagenumbering{roman}     % come?mos a numerar 

% ---------------------------------------------------------------------------- %
% Agradecimentos:
% Se o candidato n? quer fazer agradecimentos, deve simplesmente eliminar esta p?ina 
\chapter*{Agradecimentos}
Texto texto texto texto texto texto texto texto texto texto texto texto texto
texto texto texto texto texto texto texto texto texto texto texto texto texto
texto texto texto texto texto texto texto texto texto texto texto texto texto
texto texto texto texto. Texto opcional.


% ---------------------------------------------------------------------------- %
% Resumo
\chapter*{Resumo}

\noindent SOBRENOME, A. B. C. \textbf{Tíulo do trabalho em português}. 
2010. 120 f.
Tese (Doutorado) - Instituto de Matemática e Estatística,
Universidade de São Paulo, São Paulo, 2010.
\\

Elemento obrigatório, constituído de uma sequência de frases concisas e
objetivas, em forma de texto.  Deve apresentar os objetivos, métodos empregados,
resultados e conclusões.  O resumo deve ser redigido em parágrafo único, conter
no mínimo 500 palavras e ser seguido dos termos representativos do conteúdo do
trabalho (palavras-chave). 
Texto texto texto texto texto texto texto texto texto texto texto texto texto
texto texto texto texto texto texto texto texto texto texto texto texto texto
texto texto texto texto texto texto texto texto texto texto texto texto texto
texto texto texto texto texto texto texto texto texto texto texto texto texto
texto texto texto texto texto texto texto texto texto texto texto texto texto
texto texto texto texto texto texto texto texto.
Texto texto texto texto texto texto texto texto texto texto texto texto texto
texto texto texto texto texto texto texto texto texto texto texto texto texto
texto texto texto texto texto texto texto texto texto texto texto texto texto
texto texto texto texto texto texto texto texto texto texto texto texto texto
texto texto.
\\

\noindent \textbf{Palavras-chave:} palavra-chave1, palavra-chave2, palavra-chave3.

% ---------------------------------------------------------------------------- %
% Abstract
\chapter*{Abstract}
\noindent SOBRENOME, A. B. C. \textbf{Título do trabalho em inglês}. 
2010. 120 f.
Tese (Doutorado) - Instituto de Matemática e Estatística,
Universidade de São Paulo, São Paulo, 2010.
\\


Elemento obrigatório, elaborado com as mesmas características do resumo em
língua portuguesa. De acordo com o Regimento da Pós- Graduação da USP (Artigo
99), deve ser redigido em inglês para fins de divulgação. 
Text text text text text text text text text text text text text text text text
text text text text text text text text text text text text text text text text
text text text text text text text text text text text text text text text text
text text text text text text text text text text text text.
Text text text text text text text text text text text text text text text text
text text text text text text text text text text text text text text text text
text text text.
\\

\noindent \textbf{Keywords:} keyword1, keyword2, keyword3.

% ---------------------------------------------------------------------------- %
% Sum?io
\tableofcontents    % imprime o sum?io

% ---------------------------------------------------------------------------- %

% ---------------------------------------------------------------------------- %
% \chapter{Lista de Símbolos}
% \begin{tabular}{ll}
%         $\omega$    & Frequência angular\\
%         $\psi$      & Função de análise \emph{wavelet}\\
%         $\Psi$      & Transformada de Fourier de $\psi$\\
% \end{tabular}

% \clearpage
\glsaddall
\printnoidxglossaries
\clearpage

% ---------------------------------------------------------------------------- %
% Listas de figuras e tabelas criadas automaticamente
\listoffigures            
\listoftables

% ---------------------------------------------------------------------------- %
% Cap?ulos do trabalho
\mainmatter

% cabe?lho para as p?inas de todos os cap?ulos
\fancyhead[RE,LO]{\thesection}

\doublespacing              % espa?mento duplo - IPT
%\onehalfspacing            % espa?mento um e meio

\input cap-introducao        % associado ao arquivo: 'cap-introducao.tex'
\input cap-trabalhos-relacionados        % associado ao arquivo: 'cap-trabalhos-relacionados.tex'
\input cap-problema        % associado ao arquivo: 'cap-problema.tex'

\input cap-resultados-preliminares        % associado ao arquivo: 'cap-resultados-preliminares.tex'
\input cap-cronograma-quali        % associado ao arquivo: 'cap-cronograma-quali.tex'
% \input cap-conclusoes        % associado ao arquivo: 'cap-conclusoes.tex'

% cabe?lho para os ap?dices
\renewcommand{\chaptermark}[1]{\markboth{\MakeUppercase{\appendixname\ \thechapter}} {\MakeUppercase{#1}} }
\fancyhead[RE,LO]{}
\appendix

%\chapter{Sequências}
\label{ape:sequencias}

Texto texto texto texto texto texto texto texto texto texto texto texto texto
texto texto texto texto texto texto texto texto texto texto texto texto texto
texto texto texto texto texto texto.


\singlespacing

\renewcommand{\arraystretch}{0.85}
\captionsetup{margin=1.0cm}  % corre?o nas margens dos captions.
%--------------------------------------------------------------------------------------
\begin{table}
\begin{center}
\begin{small}
\begin{tabular}{|c|c|c|c|c|c|c|c|c|c|c|c|c|} 
\hline
\emph{Limiar} & 
\multicolumn{3}{c|}{MGWT} & 
\multicolumn{3}{c|}{AMI} &  
\multicolumn{3}{c|}{\emph{Spectrum} de Fourier} & 
\multicolumn{3}{c|}{Características espectrais} \\
\cline{2-4} \cline{5-7} \cline{8-10} \cline{11-13} & 
\emph{Sn} & \emph{Sp} & \emph{AC} & 
\emph{Sn} & \emph{Sp} & \emph{AC} & 
\emph{Sn} & \emph{Sp} & \emph{AC} & 
\emph{Sn} & \emph{Sp} & \emph{AC}\\ \hline \hline
 1 & 1.00 & 0.16 & 0.08 & 1.00 & 0.16 & 0.08 & 1.00 & 0.16 & 0.08 & 1.00 & 0.16 & 0.08 \\
 2 & 1.00 & 0.16 & 0.09 & 1.00 & 0.16 & 0.09 & 1.00 & 0.16 & 0.09 & 1.00 & 0.16 & 0.09 \\
 2 & 1.00 & 0.16 & 0.10 & 1.00 & 0.16 & 0.10 & 1.00 & 0.16 & 0.10 & 1.00 & 0.16 & 0.10 \\
 4 & 1.00 & 0.16 & 0.10 & 1.00 & 0.16 & 0.10 & 1.00 & 0.16 & 0.10 & 1.00 & 0.16 & 0.10 \\
 5 & 1.00 & 0.16 & 0.11 & 1.00 & 0.16 & 0.11 & 1.00 & 0.16 & 0.11 & 1.00 & 0.16 & 0.11 \\
 6 & 1.00 & 0.16 & 0.12 & 1.00 & 0.16 & 0.12 & 1.00 & 0.16 & 0.12 & 1.00 & 0.16 & 0.12 \\
 7 & 1.00 & 0.17 & 0.12 & 1.00 & 0.17 & 0.12 & 1.00 & 0.17 & 0.12 & 1.00 & 0.17 & 0.13 \\
 8 & 1.00 & 0.17 & 0.13 & 1.00 & 0.17 & 0.13 & 1.00 & 0.17 & 0.13 & 1.00 & 0.17 & 0.13 \\
 9 & 1.00 & 0.17 & 0.14 & 1.00 & 0.17 & 0.14 & 1.00 & 0.17 & 0.14 & 1.00 & 0.17 & 0.14 \\
10 & 1.00 & 0.17 & 0.15 & 1.00 & 0.17 & 0.15 & 1.00 & 0.17 & 0.15 & 1.00 & 0.17 & 0.15 \\
11 & 1.00 & 0.17 & 0.15 & 1.00 & 0.17 & 0.15 & 1.00 & 0.17 & 0.15 & 1.00 & 0.17 & 0.15 \\
12 & 1.00 & 0.18 & 0.16 & 1.00 & 0.18 & 0.16 & 1.00 & 0.18 & 0.16 & 1.00 & 0.18 & 0.16 \\
13 & 1.00 & 0.18 & 0.17 & 1.00 & 0.18 & 0.17 & 1.00 & 0.18 & 0.17 & 1.00 & 0.18 & 0.17 \\
14 & 1.00 & 0.18 & 0.17 & 1.00 & 0.18 & 0.17 & 1.00 & 0.18 & 0.17 & 1.00 & 0.18 & 0.17 \\
15 & 1.00 & 0.18 & 0.18 & 1.00 & 0.18 & 0.18 & 1.00 & 0.18 & 0.18 & 1.00 & 0.18 & 0.18 \\
16 & 1.00 & 0.18 & 0.19 & 1.00 & 0.18 & 0.19 & 1.00 & 0.18 & 0.19 & 1.00 & 0.18 & 0.19 \\
17 & 1.00 & 0.19 & 0.19 & 1.00 & 0.19 & 0.19 & 1.00 & 0.19 & 0.19 & 1.00 & 0.19 & 0.19 \\
17 & 1.00 & 0.19 & 0.20 & 1.00 & 0.19 & 0.20 & 1.00 & 0.19 & 0.20 & 1.00 & 0.19 & 0.20 \\
19 & 1.00 & 0.19 & 0.21 & 1.00 & 0.19 & 0.21 & 1.00 & 0.19 & 0.21 & 1.00 & 0.19 & 0.21 \\
20 & 1.00 & 0.19 & 0.22 & 1.00 & 0.19 & 0.22 & 1.00 & 0.19 & 0.22 & 1.00 & 0.19 & 0.22 \\ \hline 
\end{tabular}
\caption{Exemplo de tabela.}
\label{tab:tab:F5}
\end{small}
\end{center}
\end{table}

      % associado ao arquivo: 'ape-conjuntos.tex'

% ---------------------------------------------------------------------------- %
% Bibliografia
\backmatter \doublespacing   % espa?mento duplo
\bibliographystyle{plainnat-ime} % cita?o bibliogr?ica textual
\bibliography{bibliografia}  % associado ao arquivo: 'bibliografia.bib'

% ---------------------------------------------------------------------------- %
% ?dice remissivo
% \index{TBP|see{periodicidade região codificante}}
% \index{DSP|see{processamento digital de sinais}}
% \index{STFT|see{transformada de Fourier de tempo reduzido}}
% \index{DFT|see{transformada discreta de Fourier}}
% \index{Fourier!transformada|see{transformada de Fourier}}

% \printindex   % imprime o ?dice remissivo no documento 

\end{document}
