% Arquivo LaTeX de exemplo de dissertação/tese a ser apresentados ?CPG do IME-USP
% 
% Vers? 5: Sex Mar  9 18:05:40 BRT 2012
%
% Cria?o: Jes?s P. Mena-Chalco
% Revis?: Fabio Kon e Paulo Feofiloff
%  
% Obs: Leia previamente o texto do arquivo README.txt

\documentclass[12pt,twoside,a4paper]{book} % utilizando extbook para atender ao modelo do IPT

\usepackage{sectsty}
\allsectionsfont{\Large\bfseries}

% ---------------------------------------------------------------------------- %
% Pacotes 
\usepackage{fontspec}
\setmainfont{Arial} % funciona com windows/mac
%\usepackage{polyglossia}
%\setdefaultlanguage{brazil}
\usepackage[brazil]{babel}
%\usepackage[bf,small,compact]{titlesec} % cabe?lhos dos t?ulos: menores e compactos
%\usepackage[fixlanguage]{babelbib}

\usepackage[xetex]{graphicx}           % usamos arquivos pdf/png como figuras
\usepackage{setspace}                   % espa?mento flex?el
\usepackage{indentfirst}                % indenta?o do primeiro par?rafo
\usepackage{makeidx}                    % ?dice remissivo
\usepackage[nottoc]{tocbibind}          % acrescentamos a bibliografia/indice/conteudo no Table of Contents
\usepackage{courier}                    % usa o Adobe Courier no lugar de Computer Modern Typewriter
\usepackage{type1cm}                    % fontes realmente escal?eis
\usepackage{listings}                   % para formatar c?igo-fonte (ex. em Java)
\usepackage{titletoc}
\usepackage[font=small,format=plain,labelfont=bf,up,textfont=it,up]{caption}
\usepackage[usenames,svgnames,dvipsnames, table]{xcolor}
\usepackage[a4paper,top=2.54cm,bottom=2.0cm,left=2.0cm,right=2.54cm]{geometry} % margens
%\usepackage[pdftex,plainpages=false,pdfpagelabels,pagebackref,colorlinks=true,citecolor=black,linkcolor=black,urlcolor=black,filecolor=black,bookmarksopen=true]{hyperref} % links em preto



\usepackage[breaklinks,plainpages=false,pdfpagelabels,pagebackref,colorlinks=true,citecolor=DarkGreen,linkcolor=NavyBlue,urlcolor=DarkRed,filecolor=green,bookmarksopen=true]{hyperref} % links coloridos
\def\UrlBreaks{\do\/\do-}
\usepackage[all]{hypcap}                    % soluciona o problema com o hyperref e capitulos
\usepackage[round,sort]{natbib} % cita?o bibliogr?ica textual(plainnat-ime.bst)
\fontsize{60}{62}\usefont{OT1}{cmr}{m}{n}{\selectfont}


% subcaption
\usepackage{subcaption}

% todo-notes
\usepackage[colorinlistoftodos]{todonotes}


\usepackage[acronym, toc, xindy]{glossaries}
\usepackage{amssymb}
\usepackage{multirow}
\usepackage{bm}
\usepackage{stmaryrd}

\usepackage{longtable}
\usepackage[useregional]{datetime2}



\usepackage{array}
\newcolumntype{P}[1]{>{\raggedleft\arraybackslash}p{#1}}



\makenoidxglossaries
\newglossaryentry{ml}
{
    name=aprendizagem de máquina,
    description={sistema ou programa que constrói um modelo preditivo a partir de dados de entrada \citep{glossary-ml}}
}

\newglossaryentry{modelo}
{
    name=modelo,
    description={Representação do que um sistema de \gls{ml} aprendeu a partir de dados de treinamento \citep{glossary-ml}}
}

\newglossaryentry{sof}
{
    name=Stack Overflow,
    description={Site de perguntas e respostas de programação. Endereço do site: \url{https://www.stackoverflow.com/}}
}

\newglossaryentry{git}
{
    name=git,
    description={git é um versionador de controle distribuído para rastrear alterações no código-fonte durante o desenvolvimento de software. Endereço do site: \url{https://git-scm.com/} \citep{wikipedia-git-2019}}
}

\newglossaryentry{github}
{
    name=GitHub,
    description={GitHub é uma plataforma de hospedagem de código-fonte com controle de versão usando o \gls{git}. Endereço do site: \url{https://www.github.com/}}
}

\newglossaryentry{representacao-distribuida}
{
    name=representação distribuída,
    description={Representação distribuída significa uma relação de muitos para muitos entre dois tipos de representações (por exemplo, conceitos e \gls{neuron}s) \citep{Hinton-distributed-representatons:1986}
    }
}

\newglossaryentry{skip-gram}
{
    name=\textit{skip-gram},
    description={\textit{Skip-gram} é uma técnica do \gls{word2vec} que mapeia uma palavra para um vetor contínuo através da predição das palavras do contexto a partir de uma palavra-alvo. Mais informações na Seção~\ref{sec:fundamentao-representacao-tokens-palavras}
    }
}

\newglossaryentry{one-hot-encoding}
{
    name=\textit{one-hot encoding},
    description={\textit{one-hot encoding} é um vetor esparso que contém:
    \begin{itemize}
        \item Um elemento cujo valor é definido como 1
        \item O restante dos elementos tem o valor definido como 0
    \end{itemize}
    \textit{One-hot enconding} normalmente é utilizado para representar palavras e ou atributos que contém uma quantidade finita de valores \citep{glossary-ml}
    }
}

\newglossaryentry{token}
{
    name=\textit{token},
    description={\textit{token} refere-se a palavra ou termo presente nas descrições e/ou trechos de código-fonte na tarefa de recuperação de trecho de código-fonte
    }
}

\newglossaryentry{max-pooling}
{
    name=\textit{max pooling},
    description={\textit{Max pooling} refere-se a uma camada que reduz a dimensionalidade de um vetor aplicando a função \textit{max}, que retorna o maior elemento do vetor
    }
}

\newglossaryentry{neuron}
{
    name=neurônio,
    description={Um neurônio é um nó numa rede neural, que tipicamente recebe múltiplos valores de entrada e gera um valor de resultado. O neurônio aplica uma função de ativação (transformação não-linear) na soma dos valores de entrada com seus respectivos pesos \citep{glossary-ml}}
}

\newglossaryentry{bag-of-words}
{
    name=bag of words,
    description={Vetor composto por palavras indiferente à ordem e permutação. Ver exemplo na Seção~\ref{sec:representacao-das-sentencas-fundamentacao-teorica}}
}

\newglossaryentry{mecanismo-atencao}{
name=mecanismo de atenção,
description={
    O mecanismo de atenção calcula a média ponderada dos elementos de um vetor e o principal objetivo dele é encontrar um peso para cada elemento. Ao aplicarmos o mecanismo em uma tarefa de tradução, por exemplo, a cada momento que uma palavra é traduzida, o mecanismo de atenção foca em partes diferentes da sentença, i.e., ele aprende a ''prestar atenção'' nas palavras mais relevantes \citep{Goodfellow-et-al-2016}
}}

\newglossaryentry{docstring}{
name=docstring,
description={
    Em programação, um \textit{docstring} é um texto especificado no código-fonte que é usado para documentar um trecho específico do código \cite{wikipedia-docstring-2019}
}}

\newglossaryentry{jupyter}{
name=Jupyter,
description={
    \textit{Jupyter Notebook} é uma ferramenta interativa que permite desevolver, executar e documentar código em uma aplicação web. O termo \textit{notebook} refere-se a um caderno de anotações, pois é possível desenvolver, salvar as saídas do programa e fazer anotações \cite{jupyter-2019}
}}

\newglossaryentry{colab}{
name=colab,
description={
    É uma ferramenta de pesquisa e ensino para aprendizagem de máquina. É um ambiente \Gls{jupyter} que não necessita configuração ou instalação \cite{colab-2019}
}}

\newglossaryentry{unif}{
name=unif,
description={
    Arquitetura de rede neural com mecanismo de atenção proposta por \cite{cambronero-deep-learning-code-search:2019} para a recuperação de trecho de código-fonte
}}

\newglossaryentry{xlnet}{
name=XLNet,
description={
    Arquitetura autoregressiva para compreensão de linguagem durante o pré-treinamento \citep{yang2019xlNet}
}}

\newglossaryentry{word2vec}{
name=word2vec,
description={
    Word2vec é uma técnica para representar palavras através de vetores de \gls{representacao-distribuida}. Mais informações na Seção~\ref{sec:fundamentao-representacao-tokens-palavras}
}}

\newglossaryentry{tensorflow}{
name=tensorflow,
description={
    Tensorflow é uma biblioteca aberta de aprendizagem de máquina aplicável a uma ampla variedade de tarefas \citep{wikipedia-tensorflow-2020}
}}

\newglossaryentry{keras}{
name=keras,
description={
    Keras é uma biblioteca aberta de redes neurais escrita em Python \citep{wikipedia-keras-2020}
}}




\newacronym{ide}{IDE}{Integrated Development Environment}

\newacronym{rnn}{RNN}{\textit{Recurrent Neural Network}}
\newacronym{lstm}{LSTM}{\textit{Long Short Term Memory}}
\newacronym{nlp}{NLP}{\textit{Natural Language Processing}}
\newacronym{vae}{VAE}{\textit{Variational AutoEncoder}}
\newacronym{mlp}{MLP}{\textit{Multilayer Perceptron}}
\newacronym{tf-idf}{TFIDF}{\textit{Term Frequency–Inverse Document
Frequency}}
\newacronym{cbow}{CBoW}{\textit{Continuous Bag-of-Words}}
\newacronym{sof-ab}{SO}{\textit{Stack Overflow}}
\newacronym{github-ab}{GH}{\textit{GitHub}}
\newacronym{cnn}{CNN}{\textit{Convolutional Neural Network}}
\newacronym{vem}{VEM}{\textit{Workshop on Software Visualization, Evolution and Maintenance}}
\newacronym{mrr}{MRR}{\textit{Mean Reciprocal Rank}}
\newacronym{vgpu}{vGPU}{\textit{Virtual Graphics Processing Unit}}
\newacronym{elmo}{ELMo}{\textit{Embeddings from Language Models}}
\newacronym{bert}{BERT}{\textit{Bidirection Encoder Representations from Transformers}}
\newacronym{squad}{SQuAD}{\textit{Stanford Question Answering Dataset}}
\newacronym{map}{MAP}{\textit{Mean Average Precision}}
\newacronym{ndcg}{NDCG}{\textit{Normalized Discounted Cumulative Gain}}
\newacronym{gpu}{GPU}{\textit{Graphics Processing Unit}}
\newacronym{cpu}{CPU}{\textit{Central Processing Unit}}
\newacronym{nce}{NCE}{\textit{Noise Contrastive Estimation}}













% ---------------------------------------------------------------------------- %
% Cabe?lhos similares ao TAOCP de Donald E. Knuth
\usepackage{fancyhdr}
\pagestyle{fancy}
\fancyhf{}
\renewcommand{\chaptermark}[1]{\markboth{\MakeUppercase{#1}}{}}
\renewcommand{\sectionmark}[1]{\markright{\MakeUppercase{#1}}{}}
\renewcommand{\headrulewidth}{0pt}

% ---------------------------------------------------------------------------- %
\graphicspath{{./figuras/}}             % caminho das figuras (recomend?el)
\frenchspacing                          % arruma o espa?: id est (i.e.) e exempli gratia (e.g.) 
\urlstyle{same}                         % URL com o mesmo estilo do texto e n? mono-spaced
\makeindex                              % para o ?dice remissivo
\raggedbottom                           % para n? permitir espa?s extra no texto
\fontsize{60}{62}\usefont{OT1}{cmr}{m}{n}{\selectfont}
\cleardoublepage
\normalsize

% ---------------------------------------------------------------------------- %
% Op?es de listing usados para o c?igo fonte
% Ref: http://en.wikibooks.org/wiki/LaTeX/Packages/Listings
\lstset{ %
language=Java,                  % choose the language of the code
basicstyle=\footnotesize,       % the size of the fonts that are used for the code
numbers=left,                   % where to put the line-numbers
numberstyle=\footnotesize,      % the size of the fonts that are used for the line-numbers
stepnumber=1,                   % the step between two line-numbers. If it's 1 each line will be numbered
numbersep=5pt,                  % how far the line-numbers are from the code
showspaces=false,               % show spaces adding particular underscores
showstringspaces=false,         % underline spaces within strings
showtabs=false,                 % show tabs within strings adding particular underscores
frame=single,	                % adds a frame around the code
framerule=0.6pt,
tabsize=2,	                    % sets default tabsize to 2 spaces
captionpos=b,                   % sets the caption-position to bottom
breaklines=true,                % sets automatic line breaking
breakatwhitespace=false,        % sets if automatic breaks should only happen at whitespace
escapeinside={\%*}{*)},         % if you want to add a comment within your code
backgroundcolor=\color[rgb]{1.0,1.0,1.0}, % choose the background color.
rulecolor=\color[rgb]{0.8,0.8,0.8},
extendedchars=true,
xleftmargin=10pt,
xrightmargin=10pt,
framexleftmargin=10pt,
framexrightmargin=10pt
}

\usepackage{color}

\definecolor{mygreen}{rgb}{0,0.6,0}
\definecolor{mygray}{rgb}{0.5,0.5,0.5}
\definecolor{mymauve}{rgb}{0.58,0,0.82}

\lstset{ %
  language=python,
  backgroundcolor=\color{white},   % choose the background color
  basicstyle=\footnotesize,        % size of fonts used for the code
  breaklines=true,                 % automatic line breaking only at whitespace
  captionpos=b,                    % sets the caption-position to bottom
  commentstyle=\color{mygreen},    % comment style
  escapeinside={\%*}{*)},          % if you want to add LaTeX within your code
  keywordstyle=\color{blue},       % keyword style
  stringstyle=\color{mymauve},     % string literal style
  numbers=none,
}

\renewcommand{\lstlistingname}{Trecho de código-fonte}% Listing -> Algorithm
\renewcommand{\lstlistlistingname}{Lista de trechos de códigos-fontes}% List of Listings -> List of Algorithms

% ---------------------------------------------------------------------------- %


\usepackage{minted}
\usepackage[most, minted]{tcolorbox}


\newtcblisting{mypython}[1]{%
listing engine=minted,
minted style=colorful,
minted language=python,
minted options= {escapeinside=||},
listing only,
title={#1}, fonttitle=\bfseries,
enlarge top by=1cm,%     equivalent to mdframed 'skipabove'
enlarge bottom by=1cm,%  equivalent to mdframed 'skipbelow'
  }
  
\newtcblisting{mypython-linenumber}[1]{%
listing engine=minted,
minted style=colorful,
minted language=python,
minted options= {xleftmargin=10pt, escapeinside=||, linenos, breaklines},
listing only,
title={#1}, fonttitle=\bfseries,
enlarge top by=1cm,%     equivalent to mdframed 'skipabove'
enlarge bottom by=1cm,%  equivalent to mdframed 'skipbelow'
breakable=unlimited
  }
  

  
\newtcblisting{mypythonembedding}[1]{%
listing engine=minted,
colback=green!5!white,colframe=green!75!black,
minted language=python,
listing only,
fonttitle=\bfseries,
enlarge top by=1cm,%     equivalent to mdframed 'skipabove'
enlarge bottom by=1cm,%  equivalent to mdframed 'skipbelow'
minted options= {escapeinside=||},
title={#1}
}

\newtcblisting{mypythongreen}[1]{%
listing engine=minted,
colback=green!5!white,colframe=green!75!black,
minted language=python,
listing only,
fonttitle=\bfseries,
%enlarge top by=1cm,%     equivalent to mdframed 'skipabove'
%enlarge bottom by=1cm,%  equivalent to mdframed 'skipbelow'
minted options= {escapeinside=||},
title={#1}
}

\newtcblisting{mypythonred}[1]{%
listing engine=minted,
colback=red!5!white,colframe=red!75!black,
minted language=python,
listing only,
fonttitle=\bfseries,
%enlarge top by=1cm,%     equivalent to mdframed 'skipabove'
%enlarge bottom by=1cm,%  equivalent to mdframed 'skipbelow'
minted options= {escapeinside=||},
title={#1}
}

\newtcblisting{mypython-without-margin}[1]{%
listing engine=minted,
minted style=colorful,
minted language=python,
minted options= {escapeinside=||},
listing only,
title={#1}, fonttitle=\bfseries
  }


% Corpo do texto
\begin{document}
\frontmatter 
% cabe?lho para as p?inas das se?es anteriores ao cap?ulo 1 (frontmatter)
\fancyhead[RO]{{\footnotesize\rightmark}\hspace{2em}\thepage}
\setcounter{tocdepth}{2}
\fancyhead[LE]{\thepage\hspace{2em}\footnotesize{\leftmark}}
\fancyhead[RE,LO]{}
\fancyhead[RO]{{\footnotesize\rightmark}\hspace{2em}\thepage}

\onehalfspacing  % espa?mento

% ---------------------------------------------------------------------------- %
% CAPA
% Nota: O t?ulo para as disserta?es/teses do IME-USP devem caber em um 
% orif?io de 10,7cm de largura x 6,0cm de altura que h?na capa fornecida pela SPG.
\thispagestyle{empty}
\begin{center}
    \large{\textbf{Instituto de Pesquisas Tecnológicas do Estado de São Paulo}}\\
    
    \vspace*{4cm}
    
    
    
    
    \large{\textbf{Marcelo de Rezende Martins}}
    
    \vspace*{6cm}
    
    \textbf{\large{Uso de redes neurais convolucionais na recuperação de trecho de código-fonte}}\\
    
    
    
   \vspace*{10cm}
   
    \large{\textbf{São Paulo}} \\
    \large{\textbf{\the\year}}
\end{center}

% ---------------------------------------------------------------------------- %
% P?ina de rosto (S?PARA A VERS? DEPOSITADA - ANTES DA DEFESA)
% Resolu?o CoPGr 5890 (20/12/2010)
%
% IMPORTANTE:
%   Coloque um '%' em todas as linhas
%   desta p?ina antes de compilar a vers?
%   final, corrigida, do trabalho
%
%
\newpage
\thispagestyle{empty}
    \begin{center}
        Marcelo de Rezende Martins\\
        \vspace*{2.3 cm}
        Uso de redes neurais convolucionais na recuperação de trecho de código-fonte\\
        \vspace*{2 cm}
    \end{center}

    \vskip 2cm

    \hspace{6cm}\begin{minipage}{0.48\linewidth}
	Dissertação de Mestrado apresentada
	ao Instituto de Pesquisas Tecnológicas do\\
	Estado de São Paulo - IPT, como 
	parte dos requisitos para a obtenção do 
	título de Mestre em Engenharia de 
	Computação
    \end{minipage}
    
    \vskip 2cm
    
    \hspace{6cm}\begin{minipage}{0.48\linewidth}
	Data da aprovação \rule{0.7cm}{0.4pt}/\rule{0.7cm}{0.4pt}/\rule{1.4cm}{0.4pt}
    \end{minipage}
    
    \vskip 2cm
    
    \hspace{6cm}\begin{minipage}{0.48\linewidth}
	\rule{7cm}{0.4pt}\\
	Prof. Dr. Marco Aurélio Gerosa\\
	Nothern Arizona University (NAU)\\
    \end{minipage}
\\
\\
Membros da Banca Examinadora:\\ 
\\
Prof. Dr. Marco Aurélio Gerosa (Orientador)\\
Nothern Arizona University (NAU)\\
\\
Prof. Dr. Marcelo Finger (Membro)\\
Instituto de Matemática e Estatística da Universidade de São Paulo (IME-USP)\\
\\
Prof. Dr. Marcelo Novaes de Rezende (Membro)\\
Mestrado Engenharia de Computação\\


\pagebreak


% ---------------------------------------------------------------------------- %
% P?ina de rosto (S?PARA A VERS? CORRIGIDA - AP? DEFESA)
% Resolu?o CoPGr 5890 (20/12/2010)
%
% Nota: O t?ulo para as disserta?es/teses do IME-USP devem caber em um 
% orif?io de 10,7cm de largura x 6,0cm de altura que h?na capa fornecida pela SPG.
%
% IMPORTANTE:
%   Coloque um '%' em todas as linhas desta
%   p?ina antes de compilar a vers? do trabalho que ser?entregue
%   ?Comiss? Julgadora antes da defesa
%
%
\newpage
\thispagestyle{empty}
    \begin{center}
        Marcelo de Rezende Martins\\
        \vspace*{2.3 cm}
        Uso de redes neurais convolucionais na recuperação de trecho de código-fonte\\
        \vspace*{2 cm}
    \end{center}

    \vskip 2cm

    \hspace{6cm}\begin{minipage}{0.48\linewidth}
	Dissertação de Mestrado apresentada
	ao Instituto de Pesquisas Tecnológicas do\\
	Estado de São Paulo - IPT, como 
	parte dos requisitos para a obtenção do 
	título de Mestre em Engenharia de 
	Computação
    \end{minipage}
    
    \vskip 2cm
    
    \hspace{6cm}\begin{minipage}{0.48\linewidth}
	Área de Concentração: Engenharia de Software
    \end{minipage}
    
    \vskip 2cm
    
    \hspace{6cm}\begin{minipage}{0.48\linewidth}
	Orientador: Prof. Dr. Marco Aurélio Gerosa
    \end{minipage}
    
    \vskip 6cm
    
    \begin{center}
        São Paulo\\
        
        Maio / \the\year
    \end{center}

\pagebreak

\newpage
\thispagestyle{empty}

\begin{center}
        \vspace*{6 cm}
        \begin{quote}
            \textit{Sentir é criar.\\
            Sentir é pensar sem ideias, e por isso sentir é compreender, visto que o Universo não tem ideias.}
        \end{quote}
        \begin{flushright}
        Fernando Pessoa, Para Orpheu - Sentir é criar.\\
        \end{flushright}
    \end{center}


\pagebreak

\pagenumbering{roman}     % come?mos a numerar 

% ---------------------------------------------------------------------------- %
% Agradecimentos:
% Se o candidato n? quer fazer agradecimentos, deve simplesmente eliminar esta p?ina 
\chapter*{Agradecimentos}
Texto texto texto texto texto texto texto texto texto texto texto texto texto
texto texto texto texto texto texto texto texto texto texto texto texto texto
texto texto texto texto texto texto texto texto texto texto texto texto texto
texto texto texto texto. Texto opcional.


% ---------------------------------------------------------------------------- %
% Resumo
\chapter*{Resumo}

Os desenvolvedores realizam buscas por código-fonte diariamente através de buscadores de propósito geral (e.g. Google). Esses buscadores não são capazes de recuperar trechos de código-fonte semanticamente, a não ser que o código esteja acompanhado de um texto descritivo. O presente trabalho propõe uma nova abordagem de busca semântica de trechos de código-fonte através do uso de redes neurais convolucionais. As redes neurais convolucionais, que priorizam as interações locais (e.g. palavras próximas) e são invariantes a translações (e.g. deslocamento de um conjunto de palavras), mostraram-se eficazes na recuperação de trecho de código-fonte. Para verificar a eficácia da nossa proposta, comparamos o nosso modelo com dois outros métodos: um modelo de base de comparação e o atual estado da arte. Obtivemos, ao final, um resultado 5\% superior em relação ao atual estado da arte e 11\% superior ao método de comparação proposto. Além disso, o nosso modelo foi capaz de exibir o código-fonte correto nas 3 primeiras posições em 80\% das vezes e retornou na primeira posição em 60\% dos casos.
\\

\noindent \textbf{Palavras-chave:} recuperação de trecho de código-fonte, busca de código-fonte, redes neurais, redes neurais convolucionais, cnn.

% ---------------------------------------------------------------------------- %
% Abstract
\chapter*{Abstract}
\noindent \textbf{A Convolutional Neural Network approach to code retrieval}
\\

Developers search daily for code using general-purpose search engines (e.g Google). Those search engines cannot find code semantically, unless it has an accompanying description. Our work proposes a new approach to look for code semantically by using a convolutional neural network. Convolutional Neural Network (CNN), which prioritizes local interactions (e.g. words nearby) and its translation invariant (e.g shifting a set of words), showed promising results. To verify the efficacy of our approach, we compared our model with two other methods: baseline one and a state-of-the-art (SOTA). We improved SOTA by 5\% on average and the baseline one by 11\%. Our model could retrieve the correct code snippet in the top 3 (three) positions by 80\% of the time and at first position by 60\%.
\\

\noindent \textbf{Keywords:} code retrieval, code search, neural network, convolutional neural network, cnn.

% ---------------------------------------------------------------------------- %
% Sum?io
\tableofcontents    % imprime o sum?io

% ---------------------------------------------------------------------------- %

% ---------------------------------------------------------------------------- %
% \chapter{Lista de Símbolos}
% \begin{tabular}{ll}
%         $\omega$    & Frequência angular\\
%         $\psi$      & Função de análise \emph{wavelet}\\
%         $\Psi$      & Transformada de Fourier de $\psi$\\
% \end{tabular}

% \clearpage
\glsaddall
\printnoidxglossaries
\clearpage

% ---------------------------------------------------------------------------- %
% Listas de figuras e tabelas criadas automaticamente
\listoffigures            
\listoftables

% ---------------------------------------------------------------------------- %
% Cap?ulos do trabalho
\mainmatter

% cabe?lho para as p?inas de todos os cap?ulos
\fancyhead[RE,LO]{\thesection}

\doublespacing              % espa?mento duplo - IPT
%\onehalfspacing            % espa?mento um e meio

\input cap-introducao        % associado ao arquivo: 'cap-introducao.tex'
\input cap-trabalhos-relacionados        % associado ao arquivo: 'cap-trabalhos-relacionados.tex'
\input cap-problema        % associado ao arquivo: 'cap-problema.tex'
\input cap-experimento      % associado ao arquivo: 'cap-experimento.tex'
\input cap-resultados        % associado ao arquivo: 'cap-resultados.tex'
\input cap-conclusoes        % associado ao arquivo: 'cap-conclusos.tex'
% \input cap-conclusoes        % associado ao arquivo: 'cap-conclusoes.tex'

% cabe?lho para os ap?dices
\renewcommand{\chaptermark}[1]{\markboth{\MakeUppercase{\appendixname\ \thechapter}} {\MakeUppercase{#1}} }
\fancyhead[RE,LO]{}
\appendix
%% ------------------------------------------------------------------------- %%
\chapter{Ajuste dos hiper-parâmetros e normalização em lote}
\label{cap:ajuste-hiper-parametros-cnn}

Neste capítulo apresentamos os resultados dos ajustes dos hiper-parâmetros e da normalização em lote feita nas arquiteturas convolucional, \Gls{unif} e \textit{Embedding}. O intuito destes ajustes é obter um modelo mais robusto e evitar o \textit{overfitting}.

\section{Ajuste dos hiper-parâmetros da rede convolucional}
\label{sec:ajuste-hiper-parametros-cnn}

A rede convolucional exige alguns hiper-parâmetros que devem ser informados durante o treinamento. O tamanho da entrada, conforme citado no capítulo~\ref{cap:experimento}, foi fixado em 150. Caso as palavras tenham tamanho menor, o vetor é preenchido com valores $0$ ao final. Outros três parâmetros exigidos pela rede convolucional são: filtros, kernel e stride.

O parâmetro kernel define o tamanho da janela de operação de convolução. No nosso caso, que estamos utilizando a operação de convolução de 1 dimensão, Conv1D \todo{adicionar na explicação}, o kernel define os n-grams a ser extraídos do vetor de entrada. 
O parâmetro filtros define a dimensão de saída da operação de convolução. Este parâmetro indica a quantidade de filtros no resultado da operação de convolução. Cada filtro tenta extrair uma característica diferente do vetor de entrada. E o parâmetro stride define a quantidade de posições de deslocamento do filtro. O parâmetro stride foi fixado em 1. Neste caso, o filtro desloca-se por todas as posições do vetor de entrada de uma em uma posição.

Fizemos alguns testes com relação a quantidade de filtros e o tamanho do kernel utilizando como base os experimentos feitos por \cite{feng-2015} e \cite{tan-lstm-qa}. Tanto \cite{feng-2015} quanto \cite{tan-lstm-qa} utilizaram o parâmetro filtros com o valor entre 1000 e 4000. Em nossos testes, além destes valores, analisamos também os filtros para valores menores como 50, 100, 200 e 500. Já o tamanho do kernel, \cite{tan-lstm-qa} utilizou o valor 2. Em nossos testes, variamos o valor entre 2 e 4, além de concatenar kernels de diferentes valores, combinando kernels de tamanhos 2, 3, 5 e 7.

\subsection{Filtros}

Inicialmente, analisamos o comportamento da rede convolucional proposta no caítulo~\ref{cap:experimento} (ver figura XXXX\todo{adicionar figura da rede convolucional separada}) utilizando diferentes quantidades de filtros. Nas figuras a seguir, exibimos um gráfico de comparação do valor do erro no conjunto de treinamento em comparação com o erro no conjunto de validação. Conforme citado no capítulo~\ref{cap:experimento}, foram utilizadas $60.083$ amostras, sendo que $42058$ foram utilizadas para o conjunto de treinamento e $18025$ para o conjunto de validação. Neste experimento, analisamos as seguintes quantidades de filtros: 50, 100, 200, 500, 1000, 2000, 4000. Inicialmente, utilizamos o kernel com o valor 2, valor recomendado por \cite{tan-lstm-qa}.


\begin{figure}[h]
\includegraphics[width=8cm]{figuras/ape-ajustes-hiper-parametros/training-cnn-1000-k-2.png}
\caption{Gráfico do treinamento da rede convolucional na recuperação de trecho de código-fonte. Este gráfico apresenta um comparativo do erro no conjunto de validação em comparação com o erro no conjunto de treinamento. O treinamento é interrompido após 500 épocas ou caso o erro no conjunto de treinamento for menor que $1\mathbf{x}10^{-4}$. Arquitetura de rede convolucional proposta na figura XXXX do capítulo~\ref{cap:experimento}. Hiper-parâmetros: $m = 0.009$, $k = 2$}
\label{fig:ape-cnn-1000-k-2}
\end{figure}

\begin{figure}
\begin{subfigure}{.5\textwidth}
  \centering
  \includegraphics[width=.8\linewidth]{figuras/ape-ajustes-hiper-parametros/training-cnn-1000-k-2.png}
  \caption{1a}
  \label{fig:sfig1}
\end{subfigure}%
\begin{subfigure}{.5\textwidth}
  \centering
  \includegraphics[width=.8\linewidth]{figuras/ape-ajustes-hiper-parametros/training-cnn-1000-k-2.png}
  \caption{1b}
  \label{fig:sfig2}
\end{subfigure}
\begin{subfigure}{.5\textwidth}
  \centering
  \includegraphics[width=.8\linewidth]{figuras/ape-ajustes-hiper-parametros/training-cnn-1000-k-2.png}
  \caption{1c}
  \label{fig:sfig3}
\end{subfigure}
\begin{subfigure}{.5\textwidth}
  \centering
  \includegraphics[width=.8\linewidth]{figuras/ape-ajustes-hiper-parametros/training-cnn-1000-k-2.png}
  \caption{1c}
  \label{fig:sfig4}
\end{subfigure}
\caption{plots of....}
\label{fig:fig}
\end{figure}


\todo{adicionar os graficos. Rodar os testes de 50, 100, 200, 500 com filtros de tamanho 2}

\subsection{Kernel}

Conforme exibido no capítulo Arquitetura \todo{adicionar referência a arquitetura}, o kernel define na rede convolucional de 1 dimensão, o n-grams a serem extraídos. Durante o treinamento, analisamos diferentes valores para o kernel. Verificamos o comportamento para kernel de tamanho 2, 3, 4 e a combinação de valores 2, 3, 5 e 7. Conforme as figuras a seguir, não houve melhora significativa para kernels de tamanho 3 e 4 em comparação com o kernel de tamanho 2. E até mesmo a combinação de kernels não gerou uma melhora significativa dado a quantidade de parâmetros utilizados. Neste caso, um aumento da quantidade de filtros e o kernel fixado em 2 é suficiente para obter um desempenho melhor, com o valor do erro de validação mais próximo do erro de treinamento, um indicativo de modelo mais robusto.

\todo{adicionar graficos do cnn de kernel 2, 3, 4 e (2, 3, 5, 7). Somente do CNN. Para filtros 1000, 2000, 4000}





%% ------------------------------------------------------------------------- %%
\chapter{Código-fonte dos modelos}
\label{ape:codigo-fonte-dos-modelos}

Neste capítulo apresentamos os resultados dos ajustes dos hiper-parâmetros e da normalização em lote feita nas arquiteturas convolucional, \Gls{unif} e \textit{Embedding}. O intuito destes ajustes é obter um modelo mais robusto e evitar o \textit{overfitting}.

\section{Código do modelo de referência \textit{Embedding}}
\label{sec:codigo-modelo-embedding}

\begin{mypython-linenumber}{Embedding}
class EmbeddingModel(LanguageModel):
    def build(self):
        question = Input(shape=(self.question_len,), dtype='int32', name='question')
        answer = Input(shape=(self.answer_len,), dtype='int32', name='answer')

        # add embedding layers
        question_weights = np.load(self.config.initial_question_weights())
        q_embedding = Embedding(input_dim=question_weights.shape[0],
                                output_dim=question_weights.shape[1],
                                weights=[question_weights],
                                name='question_embedding')
        question_embedding = q_embedding(question)

        answer_weights = np.load(self.config.initial_answer_weights())
        a_embedding = Embedding(input_dim=answer_weights.shape[0],
                                output_dim=answer_weights.shape[1],
                                weights=[answer_weights],
                                name='answer_embedding')
        answer_embedding = a_embedding(answer)

        # maxpooling
        maxpool = Lambda(lambda x: K.max(x, axis=1, keepdims=False), output_shape=lambda x: (x[0], x[2]),
                         name='max')
        maxpool.supports_masking = True
        question_pool = maxpool(question_embedding)
        answer_pool = maxpool(answer_embedding)
        
        cos_similarity = Lambda(lambda x: cosine_similarity(x[0], x[1], axis=1)
                                       , output_shape=lambda _: (None, 1), name='similarity')([question_pool,
                                                                                               answer_pool])

        return Model(inputs=[question, answer], outputs=cos_similarity,
                                   name='qa_model')

\end{mypython-linenumber}


\section{Código do modelo Unif}


\begin{mypython-linenumber}{Camada da questão do modelo Unif}
class AverageLayer(Layer):
    def __init__(self, **kwargs):
        super(AverageLayer, self).__init__(**kwargs)

    def build(self, inputs_shape):
        inputs_shape = inputs_shape if isinstance(inputs_shape, list) else [inputs_shape]

        if len(inputs_shape) != 1:
            raise ValueError("AverageLayer expect one input.")

        # The first (and required) input is the actual input to the layer
        input_shape = inputs_shape[0]

        # Expected input shape consists of a triplet: (batch, input_length, input_dim)
        if len(input_shape) != 3:
            raise ValueError("Input shape for AverageLayer should be of 3 dimension.")

        self.input_length = int(input_shape[1])
        self.input_dim = int(input_shape[2])

        super(AverageLayer, self).build(input_shape)

    def call(self, inputs, **kwargs):
        inputs = inputs if isinstance(inputs, list) else [inputs]

        if len(inputs) != 1:
            raise ValueError("AverageLayer expect one input.")

        actual_input = inputs[0]

        # (batch, input_length, input_dim) = mean => (batch, input_dim)
        result = K.mean(actual_input, axis=1)

        return result

    def compute_output_shape(self, input_shape):
        return input_shape[0], input_shape[2] # (batch, input_dim)
\end{mypython-linenumber}


\vspace{2cm}


\begin{mypython-linenumber}{Camada de atenção do modelo Unif}
# attention from https://github.com/tech-srl/code2vec
# paper: Code2Vec: Learning Distributed Representations of Code

class AttentionLayer(Layer):
    def __init__(self, **kwargs):
        super(AttentionLayer, self).__init__(**kwargs)

    def build(self, inputs_shape):
        inputs_shape = inputs_shape if isinstance(inputs_shape, list) else [inputs_shape]

        if len(inputs_shape) < 1 or len(inputs_shape) > 2:
            raise ValueError("AttentionLayer expect one or two inputs.")

        # The first (and required) input is the actual input to the layer
        input_shape = inputs_shape[0]

        # Expected input shape consists of a triplet: (batch, input_length, input_dim)
        if len(input_shape) != 3:
            raise ValueError("Input shape for AttentionLayer should be of 3 dimension.")

        self.input_length = int(input_shape[1])
        self.input_dim = int(input_shape[2])
        attention_param_shape = (self.input_dim, 1)

        self.attention_param = self.add_weight(
            name='attention_param',
            shape=attention_param_shape,
            initializer='uniform',
            trainable=True,
            dtype=tf.float32)
        super(AttentionLayer, self).build(input_shape)

    def call(self, inputs, **kwargs):
        inputs = inputs if isinstance(inputs, list) else [inputs]

        if len(inputs) < 1 or len(inputs) > 2:
            raise ValueError("AttentionLayer expect one or two inputs.")

        actual_input = inputs[0]
        mask = inputs[1] if len(inputs) > 1 else None
        if mask is not None and not (((len(mask.shape) == 3 and mask.shape[2] == 1) or len(mask.shape) == 2)
                                     and mask.shape[1] == self.input_length):
            raise ValueError('mask should be of shape (batch, input_length)'
                             'or (batch, input_length, 1) '
                             'when calling an AttentionLayer.')

        assert actual_input.shape[-1] == self.attention_param.shape[0]

        # (batch, input_length, input_dim) * (input_dim, 1) ==> (batch, input_length, 1)
        attention_weights = K.dot(actual_input, self.attention_param)

        if mask is not None:
            if len(mask.shape) == 2:
                mask = K.expand_dims(mask, axis=2)  # (batch, input_length, 1)
            mask = K.log(mask)
            attention_weights += mask

        attention_weights = K.softmax(attention_weights, axis=1)  # (batch, input_length, 1)
        result = K.sum(actual_input * attention_weights, axis=1)  # (batch, input_length)  [multiplication uses broadcast]
        return result

    def compute_output_shape(self, input_shape):
        return input_shape[0], input_shape[2] # (batch, input_dim)
\end{mypython-linenumber}


\vspace{2cm}

\begin{mypython-linenumber}{Unif}
class UnifModel(LanguageModel):
    def build(self):
        print(tf.__version__)
        print(tf.keras.__version__)
        question = Input(shape=(self.question_len,), dtype='int32', name='question')
        answer = Input(shape=(self.answer_len,), dtype='int32', name='answer')

        # add embedding layers
        question_weights = np.load(self.config.initial_question_weights())
        q_embedding = Embedding(input_dim=question_weights.shape[0],
                                output_dim=question_weights.shape[1],
                                weights=[question_weights],
                                name='question_embedding')
        question_embedding = q_embedding(question)

        f_average_layer = AverageLayer(name='average')
        e_q = f_average_layer([question_embedding])

        answer_weights = np.load(self.config.initial_answer_weights())
        a_embedding = Embedding(input_dim=answer_weights.shape[0],
                                output_dim=answer_weights.shape[1],
                                weights=[answer_weights],
                                name='answer_embedding')
        answer_embedding = a_embedding(answer)

        f_attention_layer = AttentionLayer(name='attention')
        e_c = f_attention_layer([answer_embedding])
        
        cos_similarity = Lambda(lambda x: cosine_similarity(x[0], x[1], axis=1)
                                       , output_shape=lambda _: (None, 1), name='similarity')([e_q,
                                                                                               e_c])

        return Model(inputs=[question, answer], outputs=cos_similarity,
                                   name='qa_model')
\end{mypython-linenumber}

\section{Redes convolucionais}

\subsection{Rede convolucional com parâmetros independentes entre as camadas de questão e trecho de código-fonte}

\begin{mypython-linenumber}{Rede convolucional com parâmetros independentes}
class ConvolutionModel(LanguageModel):
    def build(self):
        assert self.config.question_len() == self.config.answer_len()

        question = Input(shape=(self.question_len,), dtype='int32', name='question')
        answer = Input(shape=(self.answer_len,), dtype='int32', name='answer')

        # add embedding layers
        question_weights = np.load(self.config.initial_question_weights())
        q_embedding = Embedding(input_dim=question_weights.shape[0],
                                output_dim=question_weights.shape[1],
                                weights=[question_weights],
                                name='question_embedding')
        question_embedding = q_embedding(question)

        answer_weights = np.load(self.config.initial_answer_weights())
        a_embedding = Embedding(input_dim=answer_weights.shape[0],
                                output_dim=answer_weights.shape[1],
                                weights=[answer_weights],
                                name='answer_embedding')
        answer_embedding = a_embedding(answer)

        # cnn
        filters = self.config.filters()
        kernel_size = self.kernel_size

        question_cnn = None
        answer_cnn = None

        if len(kernel_size) > 1:
            q_cnns = [Conv1D(kernel_size=k,
                             filters=filters,
                             activation='relu',
                             padding='same',
                             name=f'question_conv1d_{k}') for k in kernel_size]
            # question_cnn = merge([cnn(question_embedding) for cnn in cnns], mode='concat')
            question_cnn = concatenate([cnn(question_embedding) for cnn in q_cnns])
            # answer_cnn = merge([cnn(answer_embedding) for cnn in cnns], mode='concat')
            a_cnns = [Conv1D(kernel_size=k,
                             filters=filters,
                             activation='relu',
                             padding='same',
                             name=f'answer_conv1d_{k}') for k in kernel_size]
            answer_cnn = concatenate([cnn(answer_embedding) for cnn in a_cnns])
        else:
            k = kernel_size[0]
            q_cnn = Conv1D(kernel_size=k,
                             filters=filters,
                             activation='relu',
                             padding='same',
                             name=f'question_conv1d_{k}')
            # question_cnn = merge([cnn(question_embedding) for cnn in cnns], mode='concat')
            question_cnn = q_cnn(question_embedding)
            # answer_cnn = merge([cnn(answer_embedding) for cnn in cnns], mode='concat')
            a_cnn = Conv1D(kernel_size=k,
                             filters=filters,
                             activation='relu',
                             padding='same',
                             name=f'answer_conv1d_{k}')
            answer_cnn = a_cnn(answer_embedding)

        # maxpooling
        maxpool = Lambda(lambda x: K.max(x, axis=1, keepdims=False), output_shape=lambda x: (x[0], x[2]),
                         name='max')
        maxpool.supports_masking = True
        # enc = Dense(100, activation='tanh')
        # question_pool = enc(maxpool(question_cnn))
        # answer_pool = enc(maxpool(answer_cnn))
        question_pool = maxpool(question_cnn)
        answer_pool = maxpool(answer_cnn)

        cos_similarity = Lambda(lambda x: cosine_similarity(x[0], x[1], axis=1)
                                       , output_shape=lambda _: (None, 1), name='similarity')([question_pool,
                                                                                               answer_pool])

        return Model(inputs=[question, answer], outputs=cos_similarity,
                                   name='qa_model')
\end{mypython-linenumber}
\vspace{2cm}
\subsection{Rede convolucional com parâmetros compartilhados}
\begin{mypython-linenumber}{Rede convolucional com parâmetros compartilhados}
class SharedConvolutionModel(LanguageModel):
    def build(self):
        assert self.config.question_len() == self.config.answer_len()

        question = Input(shape=(self.question_len,), dtype='int32', name='question')
        answer = Input(shape=(self.answer_len,), dtype='int32', name='answer')

        # add embedding layers
        question_weights = np.load(self.config.initial_question_weights())
        q_embedding = Embedding(input_dim=question_weights.shape[0],
                                output_dim=question_weights.shape[1],
                                weights=[question_weights],
                                name='question_embedding')
        question_embedding = q_embedding(question)

        answer_weights = np.load(self.config.initial_answer_weights())
        a_embedding = Embedding(input_dim=answer_weights.shape[0],
                                output_dim=answer_weights.shape[1],
                                weights=[answer_weights],
                                name='answer_embedding')
        answer_embedding = a_embedding(answer)

        # cnn
        filters = self.config.filters()
        kernel_size = self.kernel_size

        question_cnn = None
        answer_cnn = None

        if len(kernel_size) > 1:
            cnns = [Conv1D(kernel_size=k,
                           filters=filters,
                           activation='relu',
                           padding='same',
                           name=f'shared_conv1d_{k}') for k in kernel_size]
            # question_cnn = merge([cnn(question_embedding) for cnn in cnns], mode='concat')
            question_cnn = concatenate([cnn(question_embedding) for cnn in cnns])
            # answer_cnn = merge([cnn(answer_embedding) for cnn in cnns], mode='concat')

            answer_cnn = concatenate([cnn(answer_embedding) for cnn in cnns])
        else:
            k = kernel_size[0]
            cnn = Conv1D(kernel_size=k,
                           filters=filters,
                           activation='relu',
                           padding='same',
                           name=f'shared_conv1d_{k}')
            # question_cnn = merge([cnn(question_embedding) for cnn in cnns], mode='concat')
            question_cnn = cnn(question_embedding)
            # answer_cnn = merge([cnn(answer_embedding) for cnn in cnns], mode='concat')

            answer_cnn = cnn(answer_embedding)

        # maxpooling
        maxpool = Lambda(lambda x: K.max(x, axis=1, keepdims=False), output_shape=lambda x: (x[0], x[2]),
                         name='max')
        maxpool.supports_masking = True
        # enc = Dense(100, activation='tanh')
        # question_pool = enc(maxpool(question_cnn))
        # answer_pool = enc(maxpool(answer_cnn))
        question_pool = maxpool(question_cnn)
        answer_pool = maxpool(answer_cnn)

        cos_similarity = Lambda(lambda x: cosine_similarity(x[0], x[1], axis=1)
                                       , output_shape=lambda _: (None, 1), name='similarity')([question_pool,
                                                                                               answer_pool])
        

        return Model(inputs=[question, answer], outputs=cos_similarity,
                                   name='qa_model')
\end{mypython-linenumber}

\subsection{Normalização em lote}

\subsubsection{Rede convolucional com parâmetros independentes entre as camadas}

\begin{mypython-linenumber}{Rede convolucional com parâmetros independentes}
class ConvolutionModel(LanguageModel):
    def build(self):
        assert self.config.question_len() == self.config.answer_len()

        question = Input(shape=(self.question_len,), dtype='int32', name='question')
        answer = Input(shape=(self.answer_len,), dtype='int32', name='answer')

        # add embedding layers
        question_weights = np.load(self.config.initial_question_weights())
        q_embedding = Embedding(input_dim=question_weights.shape[0],
                                output_dim=question_weights.shape[1],
                                weights=[question_weights],
                                name='question_embedding')
        question_embedding = q_embedding(question)

        answer_weights = np.load(self.config.initial_answer_weights())
        a_embedding = Embedding(input_dim=answer_weights.shape[0],
                                output_dim=answer_weights.shape[1],
                                weights=[answer_weights],
                                name='answer_embedding')
        answer_embedding = a_embedding(answer)

        # cnn
        filters = self.config.filters()
        kernel_size = self.kernel_size

        question_cnn = None
        answer_cnn = None

        if len(kernel_size) > 1:
            q_cnns = [Conv1D(kernel_size=k,
                             filters=filters,
                             activation='relu',
                             padding='same',
                             name=f'question_conv1d_{k}') for k in kernel_size]
            # question_cnn = merge([cnn(question_embedding) for cnn in cnns], mode='concat')
            question_cnn = concatenate([cnn(question_embedding) for cnn in q_cnns])
            # answer_cnn = merge([cnn(answer_embedding) for cnn in cnns], mode='concat')
            a_cnns = [Conv1D(kernel_size=k,
                             filters=filters,
                             activation='relu',
                             padding='same',
                             name=f'answer_conv1d_{k}') for k in kernel_size]
            answer_cnn = concatenate([cnn(answer_embedding) for cnn in a_cnns])
        else:
            k = kernel_size[0]
            q_cnn = Conv1D(kernel_size=k,
                             filters=filters,
                             activation='relu',
                             padding='same',
                             name=f'question_conv1d_{k}')
            # question_cnn = merge([cnn(question_embedding) for cnn in cnns], mode='concat')
            question_cnn = q_cnn(question_embedding)
            # answer_cnn = merge([cnn(answer_embedding) for cnn in cnns], mode='concat')
            a_cnn = Conv1D(kernel_size=k,
                             filters=filters,
                             activation='relu',
                             padding='same',
                             name=f'answer_conv1d_{k}')
            answer_cnn = a_cnn(answer_embedding)

        # maxpooling
        maxpool = Lambda(lambda x: K.max(x, axis=1, keepdims=False), output_shape=lambda x: (x[0], x[2]),
                         name='max')
        maxpool.supports_masking = True
        # enc = Dense(100, activation='tanh')
        # question_pool = enc(maxpool(question_cnn))
        # answer_pool = enc(maxpool(answer_cnn))
        question_pool = maxpool(question_cnn)
        answer_pool = maxpool(answer_cnn)

        cos_similarity = Lambda(lambda x: cosine_similarity(x[0], x[1], axis=1)
                                       , output_shape=lambda _: (None, 1), name='similarity')([question_pool,
                                                                                               answer_pool])

        return Model(inputs=[question, answer], outputs=cos_similarity,
                                   name='qa_model')
\end{mypython-linenumber}
\vspace{2cm}
\subsubsection{Rede convolucional com parâmetros compartilhados}

\begin{mypython-linenumber}{Rede convolucional com parâmetros compartilhados e normalização em lote}
class SharedConvolutionModelWithBatchNormalization(LanguageModel):
    def build(self):
        assert self.config.question_len() == self.config.answer_len()

        question = Input(shape=(self.question_len,), dtype='int32', name='question')
        answer = Input(shape=(self.answer_len,), dtype='int32', name='answer')

        # add embedding layers
        question_weights = np.load(self.config.initial_question_weights())
        q_embedding = Embedding(input_dim=question_weights.shape[0],
                                output_dim=question_weights.shape[1],
                                weights=[question_weights],
                                name='question_embedding')
        question_embedding = q_embedding(question)

        answer_weights = np.load(self.config.initial_answer_weights())
        a_embedding = Embedding(input_dim=answer_weights.shape[0],
                                output_dim=answer_weights.shape[1],
                                weights=[answer_weights],
                                name='answer_embedding')
        answer_embedding = a_embedding(answer)

        # cnn
        filters = self.config.filters()
        kernel_size = self.kernel_size
        question_cnn = None
        answer_cnn = None

        if len(kernel_size) > 1:
            cnns = [Conv1D(kernel_size=k,
                           filters=filters,
                           padding='same',
                           name=f'shared_conv1d_with_bn_{k}') for k in kernel_size]
            # question_cnn = merge([cnn(question_embedding) for cnn in cnns], mode='concat')

            question_outputs_cnn = [cnn(question_embedding) for cnn in cnns]
            bn_question_outputs_cnn = [BatchNormalization()(output) for output in question_outputs_cnn]
            activation_question_outputs_cnn = [Activation('relu')(output) for output in bn_question_outputs_cnn]

            question_cnn = concatenate([activation_output for activation_output in activation_question_outputs_cnn])
            # answer_cnn = merge([cnn(answer_embedding) for cnn in cnns], mode='concat')

            answer_outputs_cnn = [cnn(answer_embedding) for cnn in cnns]
            bn_answer_outputs_cnn = [BatchNormalization()(output) for output in answer_outputs_cnn]
            activation_answer_outputs_cnn = [Activation('relu')(output) for output in bn_answer_outputs_cnn]

            answer_cnn = concatenate([activation_output for activation_output in activation_answer_outputs_cnn])
        else:
            k = kernel_size[0]
            cnn = Conv1D(kernel_size=k,
                           filters=filters,
                           padding='same',
                           name=f'shared_conv1d_with_bn_{k}')
            # question_cnn = merge([cnn(question_embedding) for cnn in cnns], mode='concat')

            question_output_cnn = cnn(question_embedding)
            bn_question_output_cnn = BatchNormalization()(question_output_cnn)
            activation_question_output_cnn = Activation('relu')(bn_question_output_cnn)

            question_cnn = activation_question_output_cnn
            # answer_cnn = merge([cnn(answer_embedding) for cnn in cnns], mode='concat')

            answer_output_cnn = cnn(answer_embedding)
            bn_answer_output_cnn = BatchNormalization()(answer_output_cnn)
            activation_answer_output_cnn = Activation('relu')(bn_answer_output_cnn)

            answer_cnn = activation_answer_output_cnn

        # maxpooling
        maxpool = Lambda(lambda x: K.max(x, axis=1, keepdims=False), output_shape=lambda x: (x[0], x[2]),
                         name='max')
        maxpool.supports_masking = True
        # enc = Dense(100, activation='tanh')
        # question_pool = enc(maxpool(question_cnn))
        # answer_pool = enc(maxpool(answer_cnn))
        question_pool = maxpool(question_cnn)
        answer_pool = maxpool(answer_cnn)

        cos_similarity = Lambda(lambda x: cosine_similarity(x[0], x[1], axis=1)
                                       , output_shape=lambda _: (None, 1), name='similarity')([question_pool,
                                                                                               answer_pool])

        

        return Model(inputs=[question, answer], outputs=cos_similarity,
                                   name='qa_model')
\end{mypython-linenumber}


%\include{ape-conjuntos}      % associado ao arquivo: 'ape-conjuntos.tex'

% ---------------------------------------------------------------------------- %
% Bibliografia
\backmatter \doublespacing   % espa?mento duplo
\bibliographystyle{plainnat-ime} % cita?o bibliogr?ica textual
\bibliography{bibliografia}  % associado ao arquivo: 'bibliografia.bib'

% ---------------------------------------------------------------------------- %
% ?dice remissivo
% \index{TBP|see{periodicidade região codificante}}
% \index{DSP|see{processamento digital de sinais}}
% \index{STFT|see{transformada de Fourier de tempo reduzido}}
% \index{DFT|see{transformada discreta de Fourier}}
% \index{Fourier!transformada|see{transformada de Fourier}}

% \printindex   % imprime o ?dice remissivo no documento 

\end{document}
