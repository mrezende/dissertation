%% ------------------------------------------------------------------------- %%
\chapter{Introdução}
\label{cap:introducao}

Bilhões de linhas de código-fonte estão disponíveis hoje na internet \cite{iyer-etal-2016-summarizing}. Estas linhas de código-fonte estão disponíveis seja em repositórios de projetos \textit{open-source} ou sites e/ou fóruns da área de programação. Dado a grande quantidade de códigos-fontes disponíveis hoje, cunhou-se o termo \textit{Big Code}. \textit{Big Code} refere-se a grande quantidade de códigos-fontes a disposição atualmente, principalmente através de repositórios de código-fonte, como o Github. 

\cite{Allamanis-method-class-names:2015} extraiu informações de nomes de métodos, variáveis e classes dos principais projetos Javas no Github, por exemplo, para criar um \gls{modelo} capaz de sugerir nomes com uma maior acurácia, levando em consideração o contexto e localização do código-fonte. Já \cite{Proksch:2015} criou um modelo bayesiano de preenchimento automático (\textit{autocomplete}) que supera a maioria dos preenchimentos automáticos utilizados pelas \acrshort{ide}s, pois utiliza a informação da localização do trecho do código-fonte para fazer a sugestão do preenchimento automático. \cite{rebecca-2018} propôs um sistema de detecção automática de vulnerabilidades em código-fonte utilizando \textit{deep learning}. Milhares de métodos e funções foram extraídas dos projetos open-source para compor os dados de treinamento.

Além dos repositórios como \textit{Github} ou \textit{Bitbucket}, os fóruns de dúvida e sites com tutoriais de programação também tornaram-se fontes importantes de informação. O principal site de perguntas e respostas na área de programação atualmente é o \textit{StackOverFlow}



%% ------------------------------------------------------------------------- %%
\section{Considerações Preliminares}
\label{sec:consideracoes_preliminares}

lalalalala


%% ------------------------------------------------------------------------- %%
\section{Objetivos}
\label{sec:objetivo}

lalalalala

%% ------------------------------------------------------------------------- %%
\section{Contribuições}
\label{sec:contribucoes}

lalala

%% ------------------------------------------------------------------------- %%
\section{Organização do Trabalho}
\label{sec:organizacao_trabalho}

No Capítulo~\ref{cap:trabalhos-relacionados}, apresentamos os conceitos e trabalhos relacionados ao problema de perguntas e respostas em processamento de linguagem natural. No capítulo~\ref{cap:problema}, discutimos o problema de pesquisa proposto para o presente trabalho, a arquitetura proposta e apontamos detalhes dos dados utilizados pertinentes a pesquisa. 
No capítulo~\ref{cap:resultados-preliminares}, exibimos os resultados preliminares da arquitetura proposta utilizando os dados do \cite{yao-2018}. E também discutimos as principais dificuldades encontradas para adaptar a arquitetura do \cite{feng-2015} para resolver problema de pares de perguntas e códigos-fontes. Já no capítulo~\ref{cap:cronograma}, temos as considerações finais, proposta do cronograma e os próximos passos da pesquisa.
