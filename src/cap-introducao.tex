%% ------------------------------------------------------------------------- %%
\chapter{Introdução}
\label{cap:introducao}

O avanço e descoberta de novas tecnologias faz parte da história da humanidade. Em economia cunha-se o termo \textit{General Pupose Technology}, que são avanços tecnológicos que afetam toda uma economia. Entre estes avanços tecnológicos temos o motor a vapor, posteriormente substituído pelo motor a combustão interna. O computador representou um avanço tecnológico. A internet representou um grande salto na maneira como são feitas as comunicações e transações financeiras hoje. E atualmente, inteligência artificial está caminhando para romper esta barreira e impactar toda a economia (adicionar referências).

As primeiras redes neurais foram concebidas em 1945, o famoso perceptron. Mas desde 1945 e até os dias atuais, a área de inteligência artificial passou por vários momentos. Dentre eles momentos de euforia, no qual pesquisadores e entusiastas já previam rôbos substituindo o ser humano e que todas as tarefas seriam feitas pelos computadores. Porém demorou pelo menos 50 anos para termos os primeiros indícios do potencial desta área. Inteligência artificial abrange várias áreas. E a área que mais se destaca atualmente é a área de aprendizagem de máquina. Mais especificamente aprendizagem de máquina profunda.

Alan Turing após conceber o primeiro computador em 1930 (ver data e se está correto), disse que o computador poderia resolver qualquer problema desde que fosse possível passar as regras. O computador resolve facilmente operações aritméticas e até mesmo consegue ganhar de um campeão mundial de xadrez (1997). Porém, tarefas que são simples para o ser humano como reconhecer faces, falas ou traduzir textos, são tarefas muito complexas para o computador, pois o ser humano não consegue traduzir estas ações em regras ou fórmulas para o computador.

Porém, estes desafios de reconhecimento de imagens e tradução de textos foram superados pela máquina em 2015. Uma competição ImageNET no qual vários pesquisadores competem entre si para reconhecer milhares de imagem do banco imagenet. Um grupo do canadá apresentou o CNN, uma rede neural convolucional profunda no qual obteve 70\% de acurácia.

No ano posterior, ele obteve 97\%. Este desafio foi dado como superado. 


Dentre os fatores que os pes



%% ------------------------------------------------------------------------- %%
\section{Considerações Preliminares}
\label{sec:consideracoes_preliminares}


 Texto texto texto texto texto texto texto texto texto texto texto texto texto
texto texto texto texto texto texto texto texto texto texto texto texto texto
texto texto texto texto texto texto.

%% ------------------------------------------------------------------------- %%
\section{Objetivos}
\label{sec:objetivo}

Texto texto texto texto texto texto texto texto texto texto texto texto texto
texto texto texto texto texto texto texto texto texto texto texto texto texto
texto texto texto texto texto texto.

%% ------------------------------------------------------------------------- %%
\section{Contribuições}
\label{sec:contribucoes}

As principais contribuições deste trabalho são as seguintes:

\begin{itemize}
  \item Item 1. Texto texto texto texto texto texto texto texto texto texto
  texto texto texto texto texto texto texto texto texto texto.

  \item Item 2. Texto texto texto texto texto texto texto texto texto texto
  texto texto texto texto texto texto texto texto texto texto.

\end{itemize}

%% ------------------------------------------------------------------------- %%
\section{Organização do Trabalho}
\label{sec:organizacao_trabalho}

No Capítulo~\ref{cap:conceitos}, apresentamos os conceitos ... Finalmente, no
Capítulo~\ref{cap:conclusoes} discutimos algumas conclusões obtidas neste
trabalho. Analisamos as vantagens e desvantagens do método proposto ... 

As sequências testadas no trabalho estão disponíveis no Apêndice \ref{ape:sequencias}.
