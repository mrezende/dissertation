%% ------------------------------------------------------------------------- %%
\chapter{Introdução}
\label{cap:introducao}

Bilhões de linhas de código-fonte estão disponíveis hoje na internet \cite{iyer-etal-2016-summarizing}. Estas linhas de código-fonte estão disponíveis seja em repositórios de projetos \textit{open-source} ou sites e/ou fóruns da área de programação. Dado a grande quantidade de códigos-fontes disponíveis hoje, cunhou-se o termo \textit{Big Code}. \textit{Big Code} refere-se a grande quantidade de códigos-fontes a disposição atualmente, principalmente através de repositórios de código-fonte, como o Github. 

A partir destes repositórios, é possível extrair informações e conhecimentos. Por exemplo, há pesquisadores que coletaram padrões de código-fonte em projetos open source Java e conseguiram criar um sistema de autocompletar superior aos utilizados pelos IDEs atuais \todo{ver referencia}. Outro pesquisador\todo{ver nome} criou uma ferramenta para verificar se uma classe está no pacote correto ou não utilizando inteligência artificial. Pesquisadores da Escócia \todo{ver nome} analisaram commits no Github e criaram uma rede neural capaz de corrigir trechos de código-fonte de um programa.

E muitas vezes, estes códigos-fontes estão associados também a uma linguagem natural. No Github, o trecho de código-fonte novo enviado por um desenvolvedor, seja para acrescentar uma funcionalidade nova ou corrigir um erro, tem um comentário associado. O pesquisador XXX \todo{ver o nome}, analisou estes comentários e conseguiu criar um programa que associava um bug a um trecho de código-fonte. 



%% ------------------------------------------------------------------------- %%
\section{Considerações Preliminares}
\label{sec:consideracoes_preliminares}

lalalalala


%% ------------------------------------------------------------------------- %%
\section{Objetivos}
\label{sec:objetivo}

lalalalala

%% ------------------------------------------------------------------------- %%
\section{Contribuições}
\label{sec:contribucoes}

lalala

%% ------------------------------------------------------------------------- %%
\section{Organização do Trabalho}
\label{sec:organizacao_trabalho}

No Capítulo~\ref{cap:trabalhos-relacionados}, apresentamos os conceitos e trabalhos relacionados ao problema de perguntas e respostas em processamento de linguagem natural. No capítulo~\ref{cap:problema}, discutimos o problema de pesquisa proposto para o presente trabalho, a arquitetura proposta e apontamos detalhes dos dados utilizados pertinentes a pesquisa. 
No capítulo~\ref{cap:resultados-preliminares}, exibimos os resultados preliminares da arquitetura proposta utilizando os dados do \cite{yao-2018}. E também discutimos as principais dificuldades encontradas para adaptar a arquitetura do \cite{feng-2015} para resolver problema de pares de perguntas e códigos-fontes. Já no capítulo~\ref{cap:cronograma}, temos as considerações finais, proposta do cronograma e os próximos passos da pesquisa.
