%% ------------------------------------------------------------------------- %%
\chapter{Introdução}
\label{cap:introducao}

O software é ubíquo na sociedade moderna. Está presente praticamente em todas as atividades do cotidiano, seja no transporte, na medicina, finanças e até mesmo entretenimento, que dependem de um software de alta qualidade que esteja funcionando corretamente e seja confiável. O desenvolvimento de software é uma tarefa onerosa: desenvolvedores devem lidar com a complexidade de um software, enquanto tentam evitar erros, e ainda assim devem entregar o produto a tempo e com qualidade. Além disso, há uma demanda contínua por inovação nas ferramentas de software que ajudam a desenvolver software mais confiável e fácil de manter. Novos métodos são constantemente procurados, para reduzir a complexidade do software e ajudar os desenvolvedores a construir um software melhor (\cite{Allamanis:2018:SML}).

Conforme o software tornou-se mais presente na sociedade e com o avanço da internet, fóruns de dúvidas a respeito de programação tornaram-se cada vez mais presentes no cotidiano dos desenvolvedores. Um importante fórum de dúvidas de programação é o \Gls{sof}. Neste fórum, desenvolvedores fazem perguntas a respeito de problemas específicos de programação, algoritmos, ferramentas para desenvolvimento e práticas comuns de desenvolvimento software \cite{stackoverflow-questions-topics-2019}. 



%% ------------------------------------------------------------------------- %%
\section{Considerações Preliminares}
\label{sec:consideracoes_preliminares}

lalalalala


%% ------------------------------------------------------------------------- %%
\section{Objetivos}
\label{sec:objetivo}

lalalalala

%% ------------------------------------------------------------------------- %%
\section{Contribuições}
\label{sec:contribucoes}

lalala

%% ------------------------------------------------------------------------- %%
\section{Organização do Trabalho}
\label{sec:organizacao_trabalho}

No Capítulo~\ref{cap:trabalhos-relacionados}, apresentamos os conceitos e trabalhos relacionados ao problema de perguntas e respostas em processamento de linguagem natural. No capítulo~\ref{cap:problema}, discutimos o problema de pesquisa proposto para o presente trabalho, a arquitetura proposta e apontamos detalhes dos dados utilizados pertinentes a pesquisa. 
No capítulo~\ref{cap:resultados-preliminares}, exibimos os resultados preliminares da arquitetura proposta utilizando os dados do \cite{yao-2018}. E também discutimos as principais dificuldades encontradas para adaptar a arquitetura do \cite{feng-2015} para resolver problema de pares de perguntas e códigos-fontes. Já no capítulo~\ref{cap:cronograma}, temos as considerações finais, proposta do cronograma e os próximos passos da pesquisa.
