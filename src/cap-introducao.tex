%% ------------------------------------------------------------------------- %%
\chapter{Introdução}
\label{cap:introducao}

Dado uma questão ou descrição em linguagem natural e um conjunto de trechos de código-fonte, recuperação de trecho de código-fonte ou \textit{code retrieval} consiste em recuperar um código que solucione ou seja mais apropriado a dada descrição \citep{Allamanis-bimodal-source-code-natural-language:2015}. Podemos dizer que \textit{code retrieval} busca associar um texto em linguagem natural a um código-fonte. 

Esta associação tem diversas aplicações como busca de código-fonte a partir de uma consulta em linguagem natural, documentação e geração de programas a partir de uma especificação, por exemplo \citep{Allamanis:2018:SML}. Em engenharia de software a associação entre código-fonte e texto em linguagem natural pode auxiliar na rastreabilidade de requisitos. Além disso, pode ajudar o desenvolvedor na geração de código-fonte. Um modelo pode gerar testes de unidade a partir de uma estória de usuário \todo{citar o artigo do vem 2019, caso seja aprovado}.

O problema do \textit{code retrieval} assemelha-se ao problema de seleção de respostas ou \textit{answer selection} em \acrshort{nlp}. Este problema consiste em dado uma questão e um conjunto de possíveis respostas, ambos em linguagem natural, identificar qual resposta responde corretamente a pergunta \citep{lai-etal-2018-review}. 

A proposta deste trabalho é aplicar uma arquitetura deep learning de \textit{answer selection} no problema de \textit{code retrieval}. Inicialmente utilizaremos a arquitetura bi-LSTM com CNN proposta por \cite{tan-lstm-qa}. Além de propor uma nova arquitetura para este problema, avaliaremos o modelo utilizando os dados de entrada da base StaQC, criada por \cite{yao-2018}. Esta base de dados é composta de milhares de pares de perguntas e trechos de código-fonte do StackOverFlow \todo{citar artigo do vem 2019}.



%% ------------------------------------------------------------------------- %%
\section{Considerações Preliminares}
\label{sec:consideracoes_preliminares}

O foco deste trabalho é o problema do \textit{code retrieval}. Diferentemente dos trabalhos de \cite{iyer-etal-2016-summarizing} e \cite{Allamanis-bimodal-source-code-natural-language:2015} que abordaram também o problema de sumarização de código-fonte ou \textit{code summarization}. 

Inicialmente aplicaremos a arquitetura bi-LSTM proposta por \cite{tan-lstm-qa}. Para os resultados preliminares, iremos compará-la com outras duas arquiteturas, uma CNN com uma camada \textit{hidde layer} de entrada. E outra mais simples que tem uma camada com a representação distribuída dos dados de entrada e posteriormente uma camada \textit{maxpool} e uma função de similaridade como saída.




%% ------------------------------------------------------------------------- %%
\section{Objetivos}
\label{sec:objetivo}

lalalalala

%% ------------------------------------------------------------------------- %%
\section{Contribuições}
\label{sec:contribucoes}

lalala

%% ------------------------------------------------------------------------- %%
\section{Organização do Trabalho}
\label{sec:organizacao_trabalho}

No Capítulo~\ref{cap:trabalhos-relacionados}, apresentamos os conceitos e trabalhos relacionados ao problema de perguntas e respostas em processamento de linguagem natural. No capítulo~\ref{cap:problema}, discutimos o problema de pesquisa proposto para o presente trabalho, a arquitetura proposta e apontamos detalhes dos dados utilizados pertinentes a pesquisa. 
No capítulo~\ref{cap:resultados-preliminares}, exibimos os resultados preliminares da arquitetura proposta utilizando os dados do \cite{yao-2018}. E também discutimos as principais dificuldades encontradas para adaptar a arquitetura do \cite{feng-2015} para resolver problema de pares de perguntas e códigos-fontes. Já no capítulo~\ref{cap:cronograma}, temos as considerações finais, proposta do cronograma e os próximos passos da pesquisa.