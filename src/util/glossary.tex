\newglossaryentry{ml}
{
    name=aprendizagem de máquina,
    description={sistema ou programa que constrói um modelo preditivo a partir de dados de entrada \citep{glossary-ml}.}
}

\newglossaryentry{modelo}
{
    name=modelo,
    description={Representação do que um sistema de \gls{ml} aprendeu a partir de dados de treinamento \citep{glossary-ml}.}
}

\newglossaryentry{sof}
{
    name=stackoverflow,
    description={Site de perguntas e respostas de programação.}
}

\newglossaryentry{github}
{
    name=github,
    description={GitHub é uma plataforma de hospedagem de código-fonte com controle de versão usando o Git.}
}

\newglossaryentry{representacao-distribuida}
{
    name=representação distribuída,
    description={Representação distribuída significa uma relação de muitos para muitos entre dois tipos de representações (por exemplo, conceitos e \gls{neuron}s) \citep{Hinton-distributed-representatons:1986}. 
    \begin{itemize}
        \item Cada conceito é representado por muitos \gls{neuron}s
        \item Cada \gls{neuron} participa na representação de muitos conceitos
    \end{itemize}
    }
}

\newglossaryentry{one-hot-encoding}
{
    name=\textit{one-hot encoding},
    description={\textit{one-hot encoding} é um vetor esparso que contém:
    \begin{itemize}
        \item Um elemento cujo valor é definido como 1
        \item O restante dos elementos tem o valor definido como 0
    \end{itemize}
    \textit{One-hot enconding} normalmente é utilizado para representar palavras e ou atributos que contém uma quantidade finita de valores \citep{glossary-ml}.
    }
}

\newglossaryentry{neuron}
{
    name=neurônio,
    description={Um neurônio é um nó numa rede neural, que tipicamente recebe múltiplos valores de entrada e gera um valor de resultado. O neurônio aplica uma função de ativação (transformação não-linear) na soma dos valores de entrada com seus respectivos pesos \citep{glossary-ml}.}
}

\newglossaryentry{mecanismo-atencao}{
name=mecanismo de atenção,
description={
    Mecanismo utilizado comumente na tarefa de tradução por \gls{ml}. Este método lê uma sentença inteira de entrada e gera uma palavra traduzida por vez. E para cada momento que uma palavra é traduzida, o mecanismo de atenção foca em partes diferentes da sentença de entrada. \citep{Goodfellow-et-al-2016}.
}}


\newacronym{ide}{IDE}{Integrated Development Environment}

\newacronym{rnn}{RNN}{Recurrent Neural Network}
\newacronym{lstm}{LSTM}{Long Short Term Memory}
\newacronym{nlp}{NLP}{Natural Language Processing}
\newacronym{vae}{VAE}{Variational AutoEncoder}
\newacronym{tf-idf}{TFIDF}{Term Frequency–Inverse Document Frequency}
