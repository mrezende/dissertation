%% ------------------------------------------------------------------------- %%
\chapter{Trabalhos relacionados}
\label{cap:trabalhos-relacionados}



%% ------------------------------------------------------------------------- %%
\section{Recuperação de trecho de código-fonte ou \textit{Code retrieval}}
\label{sec:code-retrieval}

De acordo com \cite{Chen-bi-variational-autoencoder:2018}, a tarefa do \textit{code retrieval} consiste em:

\emph{Dado uma descrição em linguagem natural, recuperar o trecho de código-fonte mais relevante, tal que os desenvolvedores possam encontrar rapidamente os trechos de código que atendam as suas necessidades.}

De acordo com \cite{cambronero-deep-learning-code-search:2019}, o objetivo do \textit{code search} é recuperar um trecho de código-fonte a partir de um enorme repositório de código-fonte, que mais se aproxima da intenção do desenvolvedor, expressa em linguagem natural. 

De acordo com \cite{iyer-etal-2016-summarizing}, o problema de recuperação de trecho de código-fonte ou \textit{code retrieval} é definido como:

Seja $U_{C}$ o conjunto de trechos de código-fonte e $U_{D}$ o conjunto de descrições em linguagem natural. 

\textbf{RET:} Seja $n$ uma descrição, $n \in U_{D}$, e $C_{A}$ um conjunto trechos de código-fonte $\{c_{1}, c_{2}, ..., c_{m}\}$, $C_{A} \subset U_{C}$. Dado uma descrição $n$, o objetivo é recuperar o trecho de código-fonte $c_{j}^{*}$ do conjunto $C_{A}$, que obtenha o maior valor a partir de uma função $s$, $s \in (U_{C} X U_{D} \rightarrow \rm I\!R)$:

\begin{equation}\label{eq:code-retrieval}
c_{j}^{*} = \underset{c_{j}}{argmax} \{s(c_{j} , n)\}, 1 \leq j \leq m   
\end{equation}


Esta definição servirá de base para o problema proposto para este trabalho. Uma das primeiras referências ao problema do \textit{code retrieval} foi o trabalho de \cite{Allamanis-bimodal-source-code-natural-language:2015}. E a definição formal do problema enunciada acima na equação~\ref{eq:code-retrieval} foi proposta posteriormente por \cite{iyer-etal-2016-summarizing}. 

Há outros trabalhos \citep{Gu-deep-code-search:2018, Sachdev-neural-code-search:2018, cambronero-deep-learning-code-search:2019} que utilizam o termo \textit{code search} para o mesmo problema definido por \cite{iyer-etal-2016-summarizing}. Neste trabalho, tanto o termo \textit{code retrieval} quanto \textit{code search} irão referir-se ao mesmo problema.

\subsection{Trabalhos relacionados}\label{sec:code-retrieval-trabalhos-relacionados}

\textit{Code search} ou \textit{code retrieval} busca associar um texto em linguagem natural a um código-fonte. Esta associação tem diversas aplicações em engenharia de software conforme exposto no capítulo~\ref{cap:introducao}.

Conforme citado na seção~\ref{sec:code-retrieval}, o trabalho de \cite{Allamanis-bimodal-source-code-natural-language:2015} foi um dos primeiros sobre o problema do \textit{code retrieval}. Neste trabalho utilizou-se vetores de \gls{representacao-distribuida} tanto para as descrições quanto para os trechos de código-fonte. Uma função de classificação foi utilizada para combinar os vetores. A combinação foi feita usando operações aditivas ou multiplicativas. Ao final, o valor obtido a partir da função de classificação é transformada em probabilidade para avaliar qual trecho de código-fonte é mais relevante para uma determinada questão.

Já \cite{iyer-etal-2016-summarizing} utilizaram uma rede neural recorrente, mais especificamente \acrshort{lstm} com o \gls{mecanismo-atencao} para classificar o trecho de código-fonte de acordo com a questão. \cite{Chen-bi-variational-autoencoder:2018} utilizaram \acrfull{vae} para obter um vetor de \gls{representacao-distribuida} do código-fonte e descrição. E utilizaram algoritmos de similaridade para recuperar o trecho de código-fonte.

\cite{Gu-deep-code-search:2018} utilizaram o conceito de \textit{joint embedding}, uma técnica para juntar dados heterogêneos em um mesmo espaço vetorial tal que conceitos similares de categorias diferentes ocupem regiões próximas do espaço. Para mapear as questões e trechos de código-fonte para o mesmo espaço vetorial, \cite{Gu-deep-code-search:2018} utilizaram uma rede neural bi-\acrshort{lstm} com uma camada \textit{maxpool} ao final. Para verificar a similaridade entre os vetores, foi utilizado a função de similaridade \textit{cosine}.

\cite{Yao-coacor:2019} aprimoraram o \textit{code retrieval} combinando anotações com trechos de código-fonte. As anotações foram geradas por um modelo e aprimoradas através de aprendizagem por reforço (\textit{reinforcement learning}). Durante a aprendizagem por reforço, as anotações foram avaliadas para maximizar o sucesso na tarefa de \textit{code retrieval}.

Um grupo do facebook utilizou uma abordagem diferente. A hipótese inicial deles é que o código-fonte contém informações suficientes para ser utilizada no problema do \textit{code retrieval}. Diferente dos trabalhos citados anteriormente, eles não fizeram uso inicialmente das questões. Para cada método ou função de uma classe em um projeto de código-aberto, eles extraíram nome de método, chamadas de API, constantes e textos literais. A partir destas informações extraídas, eles criaram um vetor de representação distribuída para o método, combinando o vetor de representação distribuída de cada palavra através de uma função que é o produto entre o inverso da frequência de ocorrência da palavra no método e o TF-IFD. 

Inicialmente, o resultado obtido foi comparável a outros métodos largamente utilizados em recuperação de documentos (information retrieval). Verificou-se que os acertos ocorriam apenas quando as perguntas continham as mesmas palavras dos trechos de código-fonte. Para aprimorar, eles mapearam palavras presentes nas perguntas para palavras equivalentes no código-fonte. Isto foi feito através de um treinamento supervisionado. Após este mapeamento, eles conseguiram responder corretamente 2/3 dos pares de questões extraídos do StackOverFlow. Para eles, a resposta é considerada correta quando ela aparece entre os 10 primeiros resultados.

