%% ------------------------------------------------------------------------- %%
\chapter{Conclusões}
\label{cap:conclusoes}


A recuperação de trecho de código-fonte consiste em recuperar um trecho de código a partir de um repositório de modo a atender as intenções de um usuário, descritas em linguagem natural. Para auxiliar a recuperação de trecho de código e criar uma busca semântica, apresentamos neste trabalho uma proposta que utiliza redes neurais convolucionais na aprendizagem de representação. Os resultados, conforme apontados no Capítulo~\ref{cap:resultados}, foram promissores. A nossa arquitetura \acrshort{cnn} conseguiu classificar os trechos de código relevantes entre as 3 primeiras posições, de um total de 50, em 78\% dos casos. Além disso, conseguimos um resultado comparável com a arquitetura proposta por \cite{cambronero-deep-learning-code-search:2019}, que é o estado da arte atualmente, em nosso experimento, obtendo um resultado 5\% superior em nossa avaliação.

Além de uma nova proposta baseada em arquiteturas de seleção de respostas em NLP \citep{feng-2015, tan-lstm-qa}, avaliamos o nosso modelo em uma amostra contendo pares de questões e trechos de código-fonte em Python coletados do \Gls{sof}. A peculiaridade desta amostra é que ela contém somente questões do tipo ''how-to-do-it'', questões que exigem normalmente uma resposta direta. Além disso, diferente de \cite{cambronero-deep-learning-code-search:2019} e \cite{husain-github-semantic-search-code-2019} que tentaram criar uma busca semântica correlacionando os trechos de código a comentários \gls{docstring}, no nosso caso, buscamos correlacionar os trechos de código às questões feitas em um fórum aberto de perguntas e respostas. E conforme os próprios pesquisadores \cite{cambronero-deep-learning-code-search:2019, husain-github-semantic-search-code-2019} verificaram, as questões de fórums de dúvidas de programação aproximam-se mais das intenções de busca do usuário do que os comentários docstring. 

Ao optarmos pela arquitetura CNN, levamos em consideração as suas peculiaridades como priorização de interações locais através da extração dos n-grams mais importantes e a dificuldade em correlacionar dependências entre palavras muito distantes em uma sentença. Durante os nossos experimentos, verificamos que as redes convolucionais obtiveram um melhor desempenho através da extração de características latentes utilizando bi-grams, ao invés de combinar filtros convolucionais de diferentes tamanhos conforme apontado por \cite{tang-hybrid-deep-representation-2018} (ver Apêndice~\ref{ape:ajuste-hiper-parametros-cnn}). E quanto a dificuldade em extrair correlações entre palavras distantes, isto não foi um problema aparente dado as características de nossa amostra. A nossa amostra é composta marjoritariamente por questões e trechos de código relativamente curtos, onde 75\% da questões tem 11 palavras e 75\% dos trechos de código tem no máximo 55 palavras. 

Dado as características da arquitetura CNN e da amostra, consideramos o resultado obtido em nossos experimentos muito bom. E não esperávamos, inicialmente, um resultado tão promissor, pois conseguimos em nossa avaliação um resultado comparável com a arquitetura proposta por \cite{cambronero-deep-learning-code-search:2019}, atual estado da arte. E um ponto a favor em nossos experimentos, foi a linguagem Python, pois dado as suas características como concisa, clara, isso acabou ajudando o nosso trabalho para extração dos termos dos trechos de código e facilitou o emprego de técnicas de seleção de respostas, comumente utilizadas em NLP, na recuperação de trecho de código-fonte.


%------------------------------------------------------
\section{Considerações Finais} 
\label{sec:consideracoes-finais}

Acreditamos que a aprendizagem de representação, que mostrou-se essencial em diversas tarefas em NLP, como tradução, classificação de textos, seleção de respostas \citep{devlin-etal-2019-bert, yang2019xlNet}, é um caminho para a recuperação de trecho de código-fonte. Juntamente a isso, acreditamos que as amostras obtidas através de fóruns abertos de perguntas e respostas como o \Gls{sof} são de extrema importância, pois foram organizados e curados coletivamente pelos usuáios, tornando-se uma fonte preciosa de informações \citep{Wang-quora:2013}.

Além da aprendizagem de representação, as estratégias de seleção de respostas em NLP mostraram-se promissoras na recuperação de trecho de código-fonte. Em nosso caso, utilizamos a estratégia cunhada por \cite{feng-2015} e \cite{tan-lstm-qa}, que faz uso da aprendizagem de representação através de redes neurais convolucionais. E conforme os resultados apontados no Capítulo~\ref{cap:resultados}, o comportamento do nosso modelo na amostra contendo pares de questões e trechos código-fonte foi ao encontro dos modelos cunhados pelos pesquisadores em questões e respostas em linguagem natural. Para nós, isto é um indicativo de que as técnicas de seleção de respostas em NLP devam servir como referência e ponto de partida para futuros pesquisadores na criação de uma busca semântica de código-fonte.

Devemos salientar que os resultados obtidos até aqui são apenas um indicativo promitente do bom desempenho das estratégias e técnicas de NLP na recuperação de trecho de código-fonte. Estudos futuros para validar os resultados com usuários finais e uma análise qualitativa sobre o que as redes neurais convolucionais estão aprendendendo faz-se necessário. Além disso, utilizamos apenas 60.000 pares de questões e trechos de código-fonte durante o treinamento e 1000 pares de questões e respostas anotadas manualmente para avaliação. Esta é uma pequena porcentagem dada imensidão de dados disponíveis em fóruns abertos de perguntas e respostas, por exemplo, pois somente o StackOverFlow, até o final de abril de 2020\footnote{Consulta feita em 30 de Abril de 2020 no BigQuery do Google, que permite consultar as questões e perguntas do StackOverFlow através de uma consulta SQL}\todo{confirmar os números}, possuía mais 1.000.000 de perguntas marcadas com a palavra-chave Python. Acreditamos que somente após aprender a usar esta enorme quantidade de informações e termos criado uma base de dados organizada e curada como a ImageNet ou a Squad, teremos os primeiros avanços reais na área de recuperação de trecho de código-fonte, assim como os avanços recentes nas áreas de processamento de imagens e processamento de linguagem natural.



%------------------------------------------------------
\section{Sugestões para Pesquisas Futuras} 

Ao longo deste trabalho, levantamos algumas questões e a partir das respostas e resultados obtidos, fizemos alguns apontamentos sobre possíveis direções a serem seguidas em futuras pesquisas. Dentre elas, destacamos três importantes linhas que irão contribuir para o avanço da recuperação de trecho de código-fonte:

\begin{itemize}
    \item Conforme citado anteriormente na Seção~\ref{sec:consideracoes-finais}, a criação de uma base curada e organizada de questões e trechos de código-fonte é de suma importância para o avanço da área. E acreditamos que esta base de dados deva se basear em informações coletadas de fóruns abertos de perguntas e respostas e/ou tutoriais, no qual a descrição em linguagem natural do trecho de código aproxima-se mais da intenção de busca do desenvolvedor.
    \item O bom desempenho de estratégias já conhecidas em NLP na recuperação de trecho de código-fonte torna-se um importante ponto de partida para futuras pesquisas. E acreditamos que os recentes avanços na aprendizagem de representação obtidas pelo \acrshort{elmo}, \acrshort{bert} e \Gls{xlnet} devem permear os futuros trabalhos. 
    \item E por último e não menos importante, acreditemos que a aprendizagem de representação feita em dados coletados de fóruns abertos de perguntas e respostas, podem servir como base para treinar um modelo para recuperar trechos de código em repositórios de código-fonte como o Github. A transferência de aprendizagem, que é uma realidade em NLP e obteve ótimos resultados em diversas áreas como reconhecimento de fala, tradução e classificação de texto \cite{devlin-etal-2019-bert}, deve auxiliar novos avanços na área e ajudar na criação de uma busca semântica de código-fonte.
\end{itemize}

Estas são algumas sugestões que julgamos pertinentes e relevantes e, acreditamos, irão permear os futuros trabalhos na área de recuperação de trecho de código-fonte. E, eu, pessoalmente, tenho muito interesse em contribuir no avanço desta área e entender um pouco mais sobre como as redes neurais podem extrair a semântica do trecho de código-fonte.