%% ------------------------------------------------------------------------- %%
\chapter{Cronograma}
\label{cap:cronograma}

\section{Próximos passos}

A proposta deste trabalho é avaliar o uso das redes convolucionais no problema de \textit{code retrieval}. Conforme Bengio \todo{citar o capitulo de avaliacao do livro deep learning}, o primeiro passo é definir um objetivo, um valor alvo para o modelo. No nosso caso, o objetivo é obter um resultado comparável e, se possível, melhor que o resultado alcançado pelo modelo proposto por Sachdev, que é o estado da arte atualmente.

O processo de uso de \textit{machine learning} em um problema envolve algumas etapas. Desde o processo de tomada de decisão para uso de deep learning, coleta dos dados até a avaliação do modelo. Algumas etapas foram parcialmente concluídas durante o estudo preliminar. Os próximos passos envolvem implementar o modelo proposto por Sachdev para poder compará-lo com o nosso modelo em um mesmo conjunto de dados e ambiente de testes. De acordo com as figuras \todo{acrescentar as figuras no capítulo resultados preliminares}, a diferença entre o erro nos dados de validação e treino está alta. Acreditamos que há uma margem ainda a ser melhorada para os modelos bi-LSTM com CNN e CNN. Uma forma de remediar o \textit{overfitting} é utilizar a regularização. Para isto, avaliaremos a generalização dos modelos utilizando \textit{dropout} ou \textit{batch normalization}.

A figura abaixo ilustra as etapas do uso de \textit{machine learning}.


Coleta dos dados -> pré processamento -> treinamento -> avaliação 








