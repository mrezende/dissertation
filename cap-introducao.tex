%% ------------------------------------------------------------------------- %%
\chapter{Introdução}
\label{cap:introducao}


Problema de recuperação de trecho de código-fonte ou \textit{code retrieval} consiste em recuperar um trecho de código-fonte a partir de um repositório, que atenda as intenções do desenvolvedor, descritas em linguagem natural. Atualmente, o site do Github, que hospeda milhares de projetos de código-aberto, por exemplo, não tem um mecanismo de pesquisa que
seja capaz de recuperar um trecho de código-fonte relevante a partir dos termos utilizados no campo de busca \citep{cambronero-deep-learning-code-search:2019}.

Neste caso, o desenvolvedor acaba recorrendo a sites de dúvidas e perguntas de programação para buscar trechos de código que atenda a suas necessidades. Devido a isso, sites como o
StackOverFlow tornaram-se muito popular nos últimos anos.


Uma abordagem para este problema é encontrar um \gls{modelo} que seja capaz de correlacionar os trechos de código as intenções do usuário. Esta correlação pode ser feita através de redes neurais. 

Uma técnica comumente utilizada é a \textit{joint embedding} ou agrupamento de representações distribuídas. Esta técnica mapeia dados de diferentes distribuições, modalidades para um mesmo espaço vetorial, de tal forma que conceitos similares ocupem regiões próximas neste espaço. Técnica bastante utilizada para associar textos e imagens, traduções e até mesmo problema de perguntas e respostas e recomendações \citep{lai-etal-2018-review, Zhang:2019:deep-learning-recommender-survey}.

A representação dos dados heterogêneos na técnica joint embedding tem um papel importante. Uma boa representação deve ser capaz de auxiliar na aprendizagem de uma tarefa
posterior \citep{Goodfellow-et-al-2016:representation-learning}. No nosso caso, uma boa representação das intenções e dos trechos de código-fonte deve ser capaz de ajudar a encontrar um modelo que seja capaz de correcioná-los. 

Além disso, uma boa representação pode ser útil para dados que seguem uma outra distribuição. Uma representação de trechos de código-fonte que foram coletadas de um site de dúvidas de programação podem ser utilizadas para encontrar trechos de código-fonte similares em repositórios de código aberto.

A proposta deste trabalho é verificar o uso das redes convolucionais na aprendizagem de representação. Mais especificamente, queremos avaliar se as redes convolucionais auxiliam na obtenção de um modelo capaz de correlacionar os trechos de código-fonte as intenções do desenvolvedor.

Para isto, avaliaremos o uso das redes convolucionais no problema de \textit{code retrieval}. Avaliaremos duas arquiteturas: Uma composta somente por uma camada de rede convolucional e outra composta por uma camada de rede neural recorrente e outra camada de rede convolucional. Além de propor uma nova arquitetura para este problema, avaliaremos o modelo utilizando os dados de entrada da base StaQC, criada por \cite{yao-2018}. Esta base de dados é composta de milhares de pares de perguntas e trechos de código-fonte do StackOverFlow.



%% ------------------------------------------------------------------------- %%
\section{Considerações Preliminares}
\label{sec:consideracoes_preliminares}

O foco deste trabalho é o problema do \textit{code retrieval}. Diferentemente dos trabalhos de \cite{iyer-etal-2016-summarizing} e \cite{Allamanis-bimodal-source-code-natural-language:2015} que abordaram também o problema de sumarização de código-fonte ou \textit{code summarization}. 

Avaliaremos duas arquiteturas: Uma arquitetura bi-\acrshort{lstm} com \acrshort{cnn} proposta por \cite{tan-lstm-qa} e outra somente com camada \acrshort{cnn}. Iremos compará-las com outras outras arquiteturas de referência. A principal arquitetura de referência é a arquitetura proposta por \cite{cambronero-deep-learning-code-search:2019}, que é o estado da arte para este problema.

A base de dados disponibilizada por \cite{yao-2018} contém pares de questões e trechos de código-fonte em Python e SQL. A princípio utilizaremos apenas os pares de perguntas e código-fonte em Python.


%% ------------------------------------------------------------------------- %%
\section{Objetivos}
\label{sec:objetivo}

O objetivo principal deste trabalho é propor uma nova abordagem para o problema do \textit{code retrieval}. Queremos avaliar se o uso de redes convolucionais auxiliam na obtenção de um modelo que seja capaz de aproximar os trechos de código-fonte as intenções do desenvolvedor.

%% ------------------------------------------------------------------------- %%
\section{Contribuições}
\label{sec:contribucoes}

Dentre as contribuições deste trabalho estão:

\begin{itemize}
\item Proposta de uma nova abordagem para o problema de \textit{code retrieval};
\item Avaliação de uma nova arquitetura para o problema;
\item Avaliação do modelo utilizando uma nova base de dados disponibilizada por \cite{yao-2018};
\item Disponibilização do código-fonte, desde o pré-processamento até a avaliação final, em um repositório público;
\item Publicação dos resultados através de um artigo científico
\end{itemize}

%% ------------------------------------------------------------------------- %%
\section{Organização do Trabalho}
\label{sec:organizacao_trabalho}

No Capítulo~\ref{cap:trabalhos-relacionados}, apresentamos os conceitos e trabalhos relacionados ao problema de \textit{code retrieval}. No Capítulo~\ref{cap:abordagem}, apresentamos a nossa abordagem para o problema de pesquisa proposto. No Capítulo~\ref{cap:resultados-preliminares}, exibimos os resultados preliminares da arquitetura proposta utilizando os dados de \cite{yao-2018}. Já no capítulo~\ref{cap:cronograma} temos os próximos passos e o cronograma até a entrega do trabalho.