%% ------------------------------------------------------------------------- %%
\chapter{Introdução}
\label{cap:introducao}


Problema de recuperação de trecho de código-fonte ou \textit{code retrieval} consiste em recuperar um trecho de código-fonte a partir de um repositório, que atenda as intenções do desenvolvedor, descritas em linguagem natural \citep{cambronero-deep-learning-code-search:2019}. Podemos dizer que \textit{code retrieval} busca associar as intenções do desenvolvedor aos trechos de código-fonte.

Esta associação pode ser feita através de redes neurais. Uma abordagem é obter uma representação para a intenção e outra para os trechos de código-fonte e agrupá-las em um mesmo espaço vetorial. E o objetivo é fazer com que os vetores de representação das intenções do desenvolvedor aproximem-se dos vetores dos trechos de código-fonte que atendam as suas necessidades. 

Esta é uma abordagem comumente utilizada quando aplica-se redes neurais para encontrar uma associação em dados de diferentes modalidades e domínios. Esta abordagem é utilizada em traduções, associação de imagem e texto e, até mesmo, no problema de perguntas e respostas e recomendações em \acrshort{nlp} \citep{lai-etal-2018-review, Zhang:2019:deep-learning-recommender-survey}.

Um fator importante para a associação das representações é obter uma boa representação. Uma boa representação deve ser capaz de facilitar a aprendizagem de uma tarefa posterior \citep{Goodfellow-et-al-2016:representation-learning}. No caso de \textit{code retrieval}, uma boa representação deve ajudar na tarefa de aproximar as intenções dos trechos de código-fonte relevantes. 

Além disso, uma boa representação pode ser útil para dados que seguem uma outra distribuição. Uma representação de trechos de código-fonte que foram coletadas de um site de dúvidas de programação podem ser utilizadas para encontrar trechos de código-fonte similares em repositórios de código aberto.

A proposta deste trabalho é verificar o uso das redes convolucionais na aprendizagem de representação. Mais especificamente, queremos avaliar se as redes convolucionais auxiliam na obtenção de um modelo capaz de aproximar os trechos de código-fonte as intenções do desenvolvedor.

Para isto, avaliaremos o uso das redes convolucionais no problema de \textit{code retrieval}. Avaliaremos duas arquiteturas: Uma composta somente por uma camada de rede convolucional e outra composta por uma camada de rede neural recorrente e outra camada de rede convolucional. Além de propor uma nova arquitetura para este problema, avaliaremos o modelo utilizando os dados de entrada da base StaQC, criada por \cite{yao-2018}. Esta base de dados é composta de milhares de pares de perguntas e trechos de código-fonte do StackOverFlow.



%% ------------------------------------------------------------------------- %%
\section{Considerações Preliminares}
\label{sec:consideracoes_preliminares}

O foco deste trabalho é o problema do \textit{code retrieval}. Diferentemente dos trabalhos de \cite{iyer-etal-2016-summarizing} e \cite{Allamanis-bimodal-source-code-natural-language:2015} que abordaram também o problema de sumarização de código-fonte ou \textit{code summarization}. 

Avaliaremos duas arquiteturas: Uma arquitetura bi-\acrshort{lstm} com \acrshort{cnn} proposta por \cite{tan-lstm-qa} e outra somente com camada \acrshort{cnn}. Iremos compará-las com outras outras arquiteturas de referência e com a arquitetura proposta por \cite{cambronero-deep-learning-code-search:2019} que é o estado da arte para este problema.

A base de dados disponibilizada por \cite{yao-2018} contém pares de questões e trechos de código-fonte em Python e SQL. A princípio utilizaremos apenas os pares de perguntas e código-fonte em Python.


%% ------------------------------------------------------------------------- %%
\section{Objetivos}
\label{sec:objetivo}

O objetivo principal deste trabalho é propor uma nova abordagem para o problema do \textit{code retrieval}. Queremos avaliar se o uso de redes convolucionais auxiliam na obtenção de um modelo que seja capaz de aproximar os trechos de código-fonte as intenções do desenvolvedor.

%% ------------------------------------------------------------------------- %%
\section{Contribuições}
\label{sec:contribucoes}

Dentre as contribuições deste trabalho estão:

\begin{itemize}
\item Proposta de uma nova abordagem para o problema de \textit{code retrieval};
\item Avaliação de uma nova arquitetura para o problema;
\item Avaliação do modelo utilizando uma nova base de dados disponibilizada por \cite{yao-2018};
\item Disponibilização do código-fonte, desde o pré-processamento até a avaliação final, em um repositório público;
\end{itemize}

%% ------------------------------------------------------------------------- %%
\section{Organização do Trabalho}
\label{sec:organizacao_trabalho}

No Capítulo~\ref{cap:trabalhos-relacionados}, apresentamos os conceitos e trabalhos relacionados ao problema de \textit{code retrieval}. No Capítulo~\ref{cap:problema}, discutimos o problema de pesquisa proposto para o presente trabalho, a arquitetura proposta e mostramos detalhes dos dados utilizados pertinentes a pesquisa. 
No Capítulo~\ref{cap:resultados-preliminares}, exibimos os resultados preliminares da arquitetura proposta utilizando os dados de \cite{yao-2018}. E também discutimos as principais dificuldades encontradas para adaptar a arquitetura de \cite{tan-lstm-qa} para resolver o problema de \textit{code retrieval}. Já no capítulo~\ref{cap:cronograma}, temos as considerações finais, a proposta do cronograma e os próximos passos da pesquisa.